\documentclass{article}
\usepackage[utf8]{inputenc}
\usepackage{amsmath}

\usepackage{babel,csquotes,xpatch}
\usepackage[backend=biber]{biblatex}

\bibliography{./contribution_variables.bib}

\begin{document}

In their 2018 paper on solar photovoltaics (10.1016/j.enpol.2018.08.015; P.9 Main Body,
P.1 in Supplementary Material), the authors have a cost function C which describes
the cost associated with manufacturing one solar cell at a certain point in time. This
function depends on manufacturing variables x, y, which change over time (e.g. price
of silicon, price of chemicals, etc.). Variables are known only at discrete points in time
(t0, t1).

\begin{equation}
    C(x(t),y(t))
\end{equation}

They want to determine the contribution of a single variable x to the total change of
the cost function between two points in time $\Delta C(t0, t1)$ (eg. X\% of the cost reductions
in manufacturing solar cells in 2000 vs. 2020 are due to a change in variable x).
They start by writing out the differential of the cost function C as

\begin{equation}
    dC(x(t),y(t)) = \frac{\partial C}{\partial x} \frac{\text{d}y}{\text{d}t} \text{d}t + \frac{\partial C}{\partial y} \frac{\text{d}y}{\text{d}t} \text{d}t
\end{equation}

A sample cost model equation can be written as

\begin{equation}
    \Delta C_x = \int_{t=t_0}^{t_1} \frac{\partial C}{\partial x} \frac{\text{d}x}{\text{d}t} \text{d}t
\end{equation}

and then for C(t) assume a constant $C(t) \approx \tilde{C}$, which is approximately chosen to be $\tilde{C}=\frac{\Delta \tilde{C}}{\Delta \ln \tilde{C}}$ such that $\Delta C_x+\Delta C_y=\Delta C$

where the contribution of the change in variable x over time $t_0 < t < t_1$ is then

Here they say
If it were possible to observe the (...) variables x in continuous time, (...) [this equation]
would provide all that is needed to compute the contribution of each variable x.
Using logarithmic differentiation, they go on to rewrite the expression as

\begin{equation}
    \Delta C_x = \int_{t=t_0}^{t_1} C(t) \frac{\partial \ln C}{\partial x} \frac{\text{d}x}{\text{d}t} \text{d}t
\end{equation}

Questions:
Given a function
\begin{equation}
    C(x(t), y(t)) = x * y
\end{equation}

and discrete data for variables x(t), y(t) at the points t0, t1, what is the contribution
(total or percentage) of the change in variable x to the change $\Delta_C$

in the function C?
The authors of the paper assume that given continuous variable functions x(t), y(t), equation
(1) would suffice to calculate the contributions. But even if the time dependence
of variables were known (eg. daily data on the price of silicon, etc.), then integrating
would not yield what the authors are actually looking for.
They are interested in the contribution of single variables to the total change in C (eg.
what percentage of total manufacturing cost reductions are due to decrease in silicon
price). But integrating using

\begin{equation}
    \Delta C_x = \int_{t=t_0}^{t_1} \frac{\partial C}{\partial x} \frac{\text{d}x}{\text{d}t}
\end{equation}

is dependent on the path of curves x(t), y(t). This would yield different results for different
time dependency of variables. A variable x(t) would yield a different Cx than a
variable x'(t), which is not what the authors seek to describe.

%
\begin{equation}
    C = \frac{1}{DY}(ab+cd)
\end{equation}
%
where
%
\begin{align*}
    a &\dots \text{step 1 variable 1} \\
    b &\dots \text{step 1 variable 2} \\
    c &\dots \text{step 2 variable 1} \\
    d &\dots \text{step 2 variable 2} \\
    D &\dots \text{global variable 1} \\
    Y &\dots \text{global variable 2}
\end{align*}
%

\clearpage
\printbibliography


\end{document}
