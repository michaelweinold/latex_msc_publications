\documentclass[10pt]{article}
% page setup
\usepackage[a4paper, total={6in, 8.75in}]{geometry}
\usepackage{parskip}
% formatting
\usepackage[utf8]{inputenc} % allow utf-8 input
\usepackage[T1]{fontenc} % use 8-bit T1 fonts
% cross-referencing
\usepackage{hyperref}
\usepackage{url}
\usepackage{doi}
% tables
\usepackage{booktabs}
% fonts
\usepackage{amsfonts}
\usepackage{microtype}
% figures
\usepackage{graphicx}
\usepackage{float}
% math
\usepackage{amsmath}
% tables
\usepackage{tabularx}
\usepackage{colortbl}
% boxes
\usepackage{fancybox}
% tikz
\usepackage{tikz}
\usetikzlibrary{backgrounds}
\usetikzlibrary{arrows}
\usetikzlibrary{shapes,shapes.geometric,shapes.misc}
% landscape pages
\usepackage{pdflscape}
% chemical formulas
\usepackage{chemformula}

% this style is applied by default to any tikzpicture included via \tikzfig
\tikzstyle{tikzfig}=[baseline=-0.25em,scale=0.5]

% these are dummy properties used by TikZiT, but ignored by LaTex
\pgfkeys{/tikz/tikzit fill/.initial=0}
\pgfkeys{/tikz/tikzit draw/.initial=0}
\pgfkeys{/tikz/tikzit shape/.initial=0}
\pgfkeys{/tikz/tikzit category/.initial=0}

% standard layers used in .tikz files
\pgfdeclarelayer{edgelayer}
\pgfdeclarelayer{nodelayer}
\pgfsetlayers{background,edgelayer,nodelayer,main}

% style for blank nodes
\tikzstyle{none}=[inner sep=0mm]

% include a .tikz file
\newcommand{\tikzfig}[1]{%
{\tikzstyle{every picture}=[tikzfig]
\IfFileExists{#1.tikz}
  {\input{#1.tikz}}
  {%
    \IfFileExists{./figures/#1.tikz}
      {\input{./figures/#1.tikz}}
      {\tikz[baseline=-0.5em]{\node[draw=red,font=\color{red},fill=red!10!white] {\textit{#1}};}}%
  }}%
}

% the same as \tikzfig, but in a {center} environment
\newcommand{\ctikzfig}[1]{%
\begin{center}\rm
  \tikzfig{#1}
\end{center}}

% fix strange self-loops, which are PGF/TikZ default
\tikzstyle{every loop}=[]
%\documentclass[]{spie}  %>>> use for US letter paper
\documentclass[a4paper,nocompress]{spie}  %>>> use this instead for A4 paper
%\documentclass[nocompress]{spie}  %>>> to avoid compression of citations

\renewcommand{\baselinestretch}{1.0} % Change to 1.65 for double spacing
 
\usepackage{amsmath,amsfonts,amssymb}
\usepackage{graphicx}
\usepackage[colorlinks=true, allcolors=blue]{hyperref}
\usepackage[dvipsnames]{xcolor}
\usepackage{tabularx}
\usepackage{wrapfig}
\setlength{\extrarowheight}{2pt}
\usepackage{chemformula}

\title{Quantifying the Impact \\ of Performance Improvements and Cost Reductions \\ from 20 years of Light-Emitting Diode Manufacturing}

\author[a,b]{Michael Weinold}
\author[a]{Sergey Kolesnikov}
\author[a,c]{Laura Diaz Anadon}
\affil[a]{Centre for Environment, Energy and Natural Resource Governance, Department of Land Economy, University of Cambridge, Cambridge, CB3 9EP, UK}
\affil[b]{Chair of Entrepreneurial Risks, ETH Zurich, Scheuchzerstrasse 7, CH-8092 Zurich, CH}
\affil[c]{Belfer Center for Science and International Affairs, Harvard Kennedy School, Harvard University, Cambridge, MA 02138, USA}

\authorinfo{Further author information: (Send correspondence to Michael Weinold) \\ MW: E-mail: mw799@cam.ac.uk \\ SK: E-mail: sk2063@cam.ac.uk \\ LDA: E-mail: lda24@cam.ac.uk}

% Option to view page numbers
\pagestyle{empty} % change to \pagestyle{plain} for page numbers   
\setcounter{page}{301} % Set start page numbering at e.g. 301
 
\begin{document} 
\maketitle

\begin{abstract}
    
    With the aim of identifying and quantifying the principal sources of performance improvements and cost reductions in white light emitting diode manufacturing, we collect historical data on device cost and performance, technological breakthroughs and manufacturing innovation for phosphor-converted white light-emitting diodes for the past 20 years. We find that technological breakthroughs and process innovation contributed to performance improvements across the entire range of LED device sub-efficiencies, resulting in the overall increase in the lamp efficiency for the highest performing devices at a test current of 350mA from 5.8\% to 38.7\% between 2002 and 2020. We further develop a bottom-up manufacturing cost model with process-step resolution that captures improvements in throughput, yield and related costs of all relevant manufacturing steps, as well as economies of scale, to analyse progress in LED manufacturing cost structure between early manufacturing in 2002 and mature industry in 2020. We estimate that the cost of manufacturing low-power and mid-power light-emitting diode packages at a US location using state-of-the-art equipment has dropped from 1.11\$(2020) in 2002 to 0.05\$(2020) in 2020, a 95.5\% decrease. We also find that the largest contribution to overall cost reduction has come from increase in wafer size, and that in 2020, LED chip packaging is the largest contributor to cost at 60\%.

\end{abstract}

% Include a list of keywords after the abstract 
\keywords{light-emitting diodes, LED manufacturing, LED efficiency, manufacturing cost, cost modeling, cost reduction, technological breakthroughs}

\section{INTRODUCTION}
\label{sec:intro}


    Within just 25 years of the introduction of first commercial white light-emitting diodes (LED), the solid state lighting (SSL) industry has become a rare success story in the global drive for the transformation of energy systems. Today, the solid-state lighting markets in the US and the EU together are valued at 58.7 Bn.\$(2020) \cite{gvr2020market_us,gvr2020market_eu} with general illumination market penetration exceeding 50\% \cite{eu2019impactass,stratunl2018}. Estimates for the electrical energy saved annually from SSL adoption range from 131 TWh/year in 2020 for the EU \cite{eu2019impactass} to 442 TWh/year in 2020 for the US \cite{yamada2015adoption,guidehouse2020adoption}, which is on par with the amount of energy produced by all deployed solar photovoltaic installations in these regions. This revolution in lighting would not have been possible without dramatic improvements in the overall LED device efficiency and reductions in LED manufacturing costs.

    Understanding the sources and impact of these improvements on LED manufacturing is essential for researchers, industry professionals, as well as policymakers in the sphere of energy, science and innovation, as they may provide valuable insights for the acceleration of innovation in other demand-side energy technologies. With this aim, in this paper we identify and quantify the principal sources of performance improvements and cost reductions in phosphor-converted white LED lighting devices over the last 20 years. 

\section{Methods and Data}
\label{sec:methods}

\subsection{Choice of Metrics}
\label{subsec:metrics}

    To quantify the effect of technological breakthroughs and manufacturing process improvements, a set of metrics has to be identified that describes all dimensions of device performance, including the overall physical device efficiency and the sub-efficiencies related to different physical loss channels. They must further have the ability to directly capture the effect of individual technological breakthroughs, evolution in device architecture, and manufacturing process improvements on device performance.
    
    These requirements precluded the use of some established metrics. For instance, a highly cited and frequently updated metric is the total luminous flux per light-emitting diode package \cite{Liu2009,haitz2011solid,cho2017white,Fontoynont2018}. In combination with the cost per total flux, it is sometimes referred to as \textit{"Haitz's Law"}, in reference to an early report on LED development by Haitz et al. \cite{haitz1999case}. However, while it is often used to showcase technological progress in light-emitting diode design and manufacturing, care must be taken to consider its limitations.
    
    \begin{figure} [ht]
        \begin{center}
            \includegraphics[width=\textwidth]{haitz_law_white.pdf}
        \end{center}
        \caption{Historical increase in flux for the highest-performing white light-emitting diode chips and (multi-chip) packages, inspired by \textit{"Haitz's Law"}\cite{haitz1999case}. Shown are the datapoints for best commercial performers gathered from press releases, datasheets and industry periodicals. Note the logarithmic ordinates, black-colored datapoints corresponding to the left ordinate, and blue-colored datapoints corresponding to the right ordinate.}
        \label{fig:haitz}
    \end{figure}
    
    Firstly, today, the metric retains only limited significance as an appropriate proxy for technological progress in light-emitting diodes. This is because it is not desirable to increase the total flux per device beyond a certain point in many applications. Reasons for limiting the total flux per device may include lighting design considerations to reduce glare \cite{khan2015led}, device efficiency considerations to avoid electrical droop at high operating currents associated with high brightness \cite{Piprek2010}, and economical considerations that multiple LED chips in a single package can achieve the same brightness as a single high-brightness LED die. Secondly, the total flux per device would only be a proxy for technological improvements in light-emitting diodes if data was given for single light-emitting diode chips instead of multi-chip packages. Historically, publications have sometimes failed to make this distinction, listing datapoints for both device levels in the same graph without supporting information.

    Figure \ref{fig:haitz} shows an updated and expanded overview of the flux per device and the cost per flux of highest performing light-emitting diodes, both at the chip and package level, to demonstrate the limitations of these metrics. It is evident that the historical improvements in total flux per package for single chips are not as pronounced as for multi-chip packages.

    Instead, progress in light-emitting diode technology is best described by the overall device efficiency, or lamp efficiency. This metric is defined as the product of all device sub-efficiencies associated with an ensemble of different loss channels $\eta_L = \prod_{i=(V_f,\dots,S)} \eta_i$. Sub-efficiencies directly capture the effect of particular technological breakthroughs in device design and manufacturing process improvements. Table \ref{tab:eff} lists the relevant sub-efficiencies considered in our work, along with their mathematical definitions.
    
        \begin{table}[h!]
        \caption{List of LED device sub-efficiencies used in our methodology. We follow the definitions used by Tsao et al. \cite{tsao2010solid} and Pattison et al. \cite{pattison2017solid}. Historical developments in the sub-efficiencies are displayed in figure \ref{fig:efficiency}.}
        \bigskip
        \centering
    	\begin{tabularx}{\textwidth}{|l|l|l|X|}
    		\hline
    			\textit{Symbol} & \textit{Sub-Efficiency} & \textit{Loss-Channel} & \textit{Definition} \\
    		\hline
    		    $\eta_{V_f}$ & Forward Voltage Efficiency* & Ohmic Resistance & $\eta_{V_f} = E_{h\nu} / V_f $ \\
    		\hline
    		    $\eta_{LE}$ & Light-Extraction Efficiency & Re-absorption and Reflection & $\eta_{LE}= P_{out} / P_{in} $ \\
    		\hline
    		    $\eta_{IQ}$ & Internal Quantum Efficiency & Non-radiative Recombinations & $\eta_{EQ} = \eta_{IQ} \times \eta_{LE}$ \\
    		\hline
    		    $\eta_{Droop}$ & (Electrical) Droop & Non-radiative Recombinations & $\eta_{Droop} = 1 - \eta_{IQE} / \eta_{IQE}(A \rightarrow 0) $ \\
    		\hline
    		    $\eta_C$ & Conversion Efficiency & Stokes Loss, Absorption, etc. & $\eta_{C} = E_{\textcolor{blue}{B}} / \sum_{i=\textcolor{red}{R},\textcolor{orange}{O},\textcolor{yellow}{Y},\textcolor{teal}{G}} E_i$ \\
    		\hline
    		    $\eta_{S}$ & Spectral Efficiency & Eye Sensitivity & $\eta_{S} = K / K_{max}(CRI,CCT)$ \\
    		\hline
    		    $\eta_L$ & Lamp Efficiency & N/A (Cumulative) & $\eta_L = \prod_{i=(V_f,\dots,S)} \eta_i$ \\
            \hline
                \multicolumn{4}{|l|}{$\!\begin{aligned}
                    E_{h\nu} &\dots \text{photon energy} \\
                    V_f &\dots \text{forward voltage} \\
                    A &\dots \text{electrical current} \\
                    E_{B,\dots,G} &\dots \text{optical energy of monochromatic light (blue, red, orange, yellow, green)} \\
                    K &\dots \text{luminous efficacy of radiation} \\
                    CRI &\dots \text{color rendering index}, \ CCT \dots \text{color temperature} \\
                \end{aligned}$} \\
            \hline
    	\end{tabularx}
    	\label{tab:eff}
    \end{table}
    
    Performance improvements in metrics related to consumer experience have also played a role in the adoption of LED-based luminaires \cite{cowan2011understanding}. Broadly described as the quality of light, these include metrics related to the emitted spectrum, as well as flicker, the temporal modulation of light. For instance, breakthroughs in the development of down-conversion phosphors enabled a greater range in the color temperature of light sources. The analysis of consumer experience metrics is ongoing and will be presented in our future research. Notably, we exclude flicker from this work because it is an inherent property not of the light-emitting diodes themselves, but rather the electrical ballasts in the luminaires. For further discussion of flicker in this context, see a recent publication by Weinold \cite{weinold2020long}.

    \subsection{Data Sources and Performance Calculations}
    \label{subsec:data}
    
        To gain a detailed understanding of the sources and magnitude of efficiency improvements and cost reductions in LED manufacturing, we gathered historical data on device cost and performance across different sub-efficiencies, as well as on technological breakthroughs and manufacturing innovation for phosphor-converted white light-emitting diodes for the past 20 years. The data was collected from a variety of sources such as patent literature, scientific articles, company publications, industry reports and roadmaps, and technical periodicals. Additional information was provided by experts from academia, the LED manufacturing industry, and the manufacturing equipment industry during semi-structured interviews. We further had to identify the associated device architectures, manufacturing processes and types of down-conversion phosphors used. 
        
        \begin{figure} [ht]
            \begin{center}
                \includegraphics[width=0.85\textwidth]{SPIE/article/droop_lumileds.pdf}
            \end{center}
            \caption{Luminous intensity of four different \textit{Lumileds} high-power light-emitting diodes normalized to the value at a test current of $A_{test}=350$mA. The black curves describe the real measured intensity, the orange curves describe the estimated ideal intensity. Droop $D$, as defined in table \ref{tab:eff}, is the difference between these curves at the test current. Current-Intensity data extracted from device datasheets \cite{datasheet_lumileds_lux1,datasheet_lumileds_rebel,datasheet_lumileds_rebplus,lumi2019data}}
            \label{fig:droop}
        \end{figure}
        
        In cases where efficiency data was not readily available, we calculated the respective values from raw performance data. For instance, to calculate historical data on the reduction of losses associated with electrical droop, we used device performance data extracted from datasheets by the method shown in figure \ref{fig:droop}. This data was cross-referenced with information on the different chip architectures and manufacturing processes used for each device. In a similar way, data was gathered for all sub-efficiencies listed in table \ref{tab:eff} to gain a complete picture of the historical developments in light-emitting diode efficiency.

    \subsection{Manufacturing Cost Model}
    \label{subsec:costmodel}
    

        To quantify changes in the manufacturing cost of LED devices, a bottom-up manufacturing cost model with process step resolution was constructed. It covers the entire manufacturing process of GaN-on-sapphire-based phosphor-converted low-to-mid power light-emitting diode packages of different chip architectures. We designed it to accommodate the classical p-side-up lateral current spreading architecture, with two further architectures\textemdash a packaged flip-chip vertical current spreading architecture and a chip-scale package flip chip architecture\textemdash under development. An excerpt of the evolution of the classical chip architecture considered in the cost model is shown in figure \ref{fig:chip_arch}. The earliest commercial warm-white phosphor converted blue light emitting diodes were introduced in 2002, so the model was constructed for the years 2002, 2012 and 2020. It was populated with equipment data from European and North American firms, selected for a virtual North American manufacturing location. Details of the manufacturing processes and process-specific step parameters were derived from the sources mentioned in section \ref{subsec:data}. For 2012, additional data for the model was adapted from the \textit{LEDCOM} cost model prepared for the US Department of Energy by Stephen Bland of SB Consulting\cite{ledcomv2}. Our model includes the wafer treatment process with further ongoing work also focusing on the chip packaging process. While the model offers great flexibility in adapting the manufacturing process parameters and chip architectures used, it is important to note the limitations of this approach. The main aim of the model is not to faithfully represent real-world manufacturing such as in present-day manufacturing locations in Asia, but rather to show the effect of technological change and manufacturing process improvements on total cost. 

        \begin{figure} [ht]
            \begin{center}
                \includegraphics[width=\textwidth]{SPIE/article/chip_architectures.pdf}
            \end{center}
            \caption{Cutaway side views of the evolution of chip architectures for classical chip designs (lateral current spreading). Note that dimensions are not to scale. Years correspond to earliest identified patent priority date. Illustration in the box indicates a chip design not brought to large scale production. Drawings adapted from patents \cite{nagahama2013nitride,tanaka2010semiconductor,wierer2006photonic}}
            \label{fig:chip_arch}
        \end{figure}

        To disaggregate the contribution of changes in single variables to changes in the total manufacturing cost, we used an approach introduced by Kavlak and colleagues in a cost model for photovoltaic modules \cite{kavlak2018evaluating}. It is based on the logarithmic derivative of the total differential of the cost function. For the detailed derivation, we refer to the supplementary material of the original publication.

\section{RESULTS}

\subsection{Performance Improvements}

     Main results for performance improvements are presented in Figure \ref{fig:efficiency}. Overall efficiency in best performing devices improved from $\eta_L=5.8\%$ in 2002 to $\eta_L = 38.7\%$ in 2020. As previous studies have noted, no single loss channel dominates the overall the efficiency\cite{tsao2010solid}. Figure \ref{fig:efficiency} additionally shows the physical limits for the loss channels. We find that those loss channels with a fixed physical limit below 100\% have become significantly more dominant in 2016 and 2020. An example of this tendency is the increased importance of Stokes loss that describes the energy dissipated upon conversion from short wavelength to long wavelength photons. Notably, sub-efficiencies for the most current devices are only $\sim10\%$ below the physical limit. The exception is spectral efficiency, which at $\sim17\%$ below the physical limit shows larger potential for improvement.
     
     Our comparison between the improvements in device sub-efficiencies between 2002 and 2020 shows that the aggregate efficiency improvement was not driven primarily by improvements in a single sub-efficiency. Instead, there has been consistent progress across the ensemble of loss channels corresponding to the device sub-efficiencies in the past 18 years, including forward voltage efficiency ($70\%\rightarrow99.5\%$), internal quantum efficiency ($55\%\rightarrow90\%)$, electrical droop ($65\%\rightarrow90\%$), light-extraction efficiency ($60\%\rightarrow90\%$) and spectral efficiency ($74\% \rightarrow83\%$).
     
    \begin{figure} [ht]
        \begin{center}
            \includegraphics[width=0.85\textwidth]{SPIE/article/breakthroughs_efficiency.pdf}
        \end{center}
        \caption{Historical changes in sub-efficiencies of phosphor-converted warm white light-emitting diodes with test currents of at least $I_\text{test}=350$mA. The overall lamp efficiency $\eta_L$ is displayed as the rightmost column. This figure takes as inputs the state-of-the-art sub-efficiencies discussed in section \ref{subsec:metrics}. Horizontal colored bars give state-of-the-art sub-efficiencies for five years: \textcolor{blue}{1997}, \textcolor{teal}{2002}, \textcolor{orange}{2010}, \textcolor{magenta}{2016} and \textcolor{red}{2020}. Colored annotation "N/A" indicates that the sub-efficiency of the corresponding year cannot be computed for the following reasons: V$_\text{f}$E, Droop: depend on current, which was below 350mA at the time; CE, SE: warm white spectrum LEDs not available at the time. Physical limits are indicated by black horizontal bars. Possible range for the physical limit of V$_\text{f}$E exceeds 100\%, depends on electrical device parameters, and is indicated by an upward pointing black arrow. Efficiency acronyms are listed in table \ref{tab:eff}.}
        \label{fig:efficiency}
    \end{figure}
    
    Notably, improvements in sub-efficiencies with a good understanding of the underlying physical loss channels, such as light-extraction efficiency, forward voltage efficiency and spectral efficiency, were mostly a result of targeted research and development. For instance, optical simulations could be performed to determine ways of modifying the device architecture to improve light extraction efficiency. On the other hand, efficiency improvements in sub-efficiencies with a relatively limited understanding of underlying physical mechanisms, such as internal quantum efficiency and the associated electrical droop at high currents, were mostly due to manufacturing process improvements. For instance, the reasons for electrical droop have, until recently, been a topic of hot debate. Despite that, significant improvements in this channel were still achieved by varying metal-organic vapour deposition (MOCVD) growth parameters in large ensembles of wafer batches.

\subsection{Manufacturing Cost Reductions}
    
    Results of manufacturing cost modeling for three reference years (2002, 2012, and 2020), broken down by five main cost components in our model\textemdash substrate, epitaxy, wafer processing, chip packaging, and phosphor\textemdash are presented in table \ref{tab:cost}. Using our cost model, we estimate that the cost of manufacturing low-to-mid-power light-emitting diode packages at a US location, using state-of-the-art equipment, have declined from 1.11\$(2020) in 2002 to 0.05\$(2020) in 2020, a 95.5\% overall cost reduction. 
    
    A preliminary analysis of the cost model components suggests that an increase in the wafer size is responsible for the largest part of the overall cost reduction. The wafer diameter $d$ used in production has been steadily increasing since 2002. The model assumes diameters and associated number of die per wafer changing from $d(2002)=50$mm$\rightarrow851$ to $d(2020)=200$mm$\rightarrow26,838$. Related decreases in exclusion zone and cutting width have also contributed to the increase in die per wafer number. Further analysis of the relative contributions of different manufacturing process steps to the cost structure suggests that as the number of die per wafer increases, the packaging steps now carry the largest share of the total cost at 60\% in 2020. This is because the output of these steps is limited by the throughput of the associated equipment. The wafer processing steps, which made the largest contribution to the total cost in 2002, depend on the number of die per wafer and only to a lesser extent on the throughput of equipment, which explains why they became relatively less important in the cost structure by 2020.
    
    \begin{table}[h!]
        \caption{Manufacturing cost of low-power classical chips (lateral current spreading) estimated with the cost model introduced in section \ref{subsec:costmodel}.}
        \bigskip
            \centering
            \begin{tabularx}{\textwidth}{|l|l|X|X|X|X|X|l|}
            	\hline
            		\textit{Year} & \textit{Unit} & \textit{Substrate} & \textit{Epitaxy} & \textit{Wafer Proc.} & \textit{Packaging} & \textit{Phosphor} & \textit{Total} \\
                \hline
                    2002 & \$(2020) & 0.0760 & 0.1638 & 0.7351 & 0.1203 & 0.0175 & 1.1131 \\
                \hline
                    2002-2012 & $\Delta$ \% & -98.8 & -94.7 & -96.2 & -47.8 & -42.2 & -90.1 \\
                \hline
                    2012 & \$(2020) & 0.0008 & 0.0086 & 0.0276 & 0.0627 & 0.0101 & 0.0111 \\
                \hline
                    2012-2020 & $\Delta$ \% & -81.8 & -34.37 & -61.5 & -50.3 & -56.4 & -52.7 \\
                \hline
                    2020 & \$(2020) & 0.0001 & 0.0056 & 0.0106 & 0.0311 & 0.0044 & 0.0528 \\
                \hline
            \end{tabularx}
            \label{tab:cost}
        \end{table}


\section{CONCLUSIONS}

    In this study, we found that technological breakthroughs and manufacturing process innovation contributed to performance improvements across the entire range of LED device sub-efficiencies, resulting in the overall lamp efficiency increase for the highest performing devices at a test current of 350mA from 5.8\% in 2002 to 38.7\% in 2020. Notably, we found that efficiency improvements in loss channels with a good understanding of underlying physical mechanisms were driven mostly by research and development, while improvements in channels with a lack of such understanding required extensive experimentation and learning-by-doing in the manufacturing setting. It shows the importance of further research into the underlying solid-state physics of light-emitting diodes. A deeper understanding of the effects and mechanisms associated with device loss channels will potentially enable further advances in LED technology, with spectral efficiency in particular showing promise for further improvements. 

    We also constructed the manufacturing cost model for low-to-mid-power LED packages produced at a virtual US location using state-of-the-art equipment. We estimated that the cost of LED manufacturing in these conditions decreased by 95.5\% from 1.11\$(2020) in 2002 to 0.05\$(2020) in 2020. The largest contribution to the overall cost reductions came from increase in the wafer size, while the largest remaining contributor to the manufacturing cost is the LED packaging, which accounts for 60\% of the cost structure in 2020.

\acknowledgments % equivalent to \section*{ACKNOWLEDGMENTS}       

We would like to thank Gabriel Chan, Anna Goldstein and Venkatesh Narayanamurti for many helpful discussions and their feedback. We also express our deep gratitude to all interviewees for their willingness to participate in this study and invaluable contributions. This research is funded by the \textit{Alfred P. Sloan Foundation}, grant number 253128. Michael Weinold further gratefully acknowledges support from the \textit{Swiss Study Foundation} of Zurich, Switzerland.

\clearpage
% References
\bibliography{report} % bibliography data in report.bib
\bibliographystyle{spiebib} % makes bibtex use spiebib.bst

\end{document} 


%%% custom definitions %%%%%%%%%%%%%%%%%%%%%%%%%%%%%%%%%%%%%%%%%%%%%%%%%%%%%%%%%%%%%%%%%

\definecolor{silver}{rgb}{0.75, 0.75, 0.75}

%%% document metadata %%%%%%%%%%%%%%%%%%%%%%%%%%%%%%%%%%%%%%%%%%%%%%%%%%%%%%%%%%%%%%%%%%

\title{Rapid technological progress in white light-emitting diodes \\ and its sources in innovation and technology spillovers  }
\date{February 2023}

\hypersetup{
    pdftitle={Supplementary Information},
    pdfauthor={Michael Weinold, Sergey Kolesnikov, Laura Diaz Anadon}
}

%%% document body %%%%%%%%%%%%%%%%%%%%%%%%%%%%%%%%%%%%%%%%%%%%%%%%%%%%%%%%%%%%%%%%%%%%%%%

\begin{document}

\setlength{\fboxsep}{10pt}
\fbox{
    \parbox{\textwidth}{
        \textbf{\textsc{Supplementary Information}} for: \\
        M. Weinold, S. Kolesnikov, L.D. Anadon \\
        "Rapid technological progress in white light-emitting diodes and its sources in innovation and technology spillovers" \textit{Energy and Environmental Science} (2023)
    }
}

\tableofcontents

\newpage

\section{Lighting Metrics}

\subsection{Luminous Efficacy (of Radiation)}
\label{subsec:ler}

Efficacy in lighting is dependent on the luminosity function, which describes the wavelength-dependent sensitivity of the human eye. A light source emitting very \textit{efficiently} in the infrared yet emitting no visible light has a very low \textit{efficacy}. The luminous efficacy of radiation $K$ is mathematically defined as the normalized, integrated product of the spectral radiant flux of a light source with the wavelength dependent human sensitivity to light \cite{cie-term-effrad}

\begin{equation}
\label{eqn:ler}
    K [\text{lm/W}_{opt}]= \frac{\int_0^\infty K( \lambda ) \phi \text{d} \lambda}{\int_0^\infty \phi \text{d} \lambda}
\end{equation}

where

\begin{align*}
    K &\dots \text{spectral luminous efficacy} \\
    \phi &\dots \text{spectral radiant flux} \\
    \lambda &\dots \text{wavelength}
\end{align*}

This metric can be computed from spectral data alone and does not require additional spectral normalization. It enables straightforward comparison between the performance of different downconversion phosphors, as shown in figure \ref{fig:se_efficacy} in chapter \ref{ch:results}. Light sources emitting in the far red or blue part of the spectrum have lower efficacy of radiation, as the product with the luminosity function of figure \ref{fig:tristimulus} shows. Care must be taken not to confuse this efficacy metric with efficacy of source, which depends not only on the spectral power distribution of a light source.

This metric describes the match of a light-emitting diode package spectrum to the human visual system. It thus describes only a part of the total device performance. 

\subsection{Luminous Efficacy (of Source)}
\label{subsec:les}

In contrast to the efficacy \textit{of radiation} $K$, the luminous efficacy \textit{of a light source} $\eta$ is defined as the ratio between the emitted luminous flux and the consumed electrical power \cite{cie-term-effsrc}

\begin{equation}
    \eta [\text{lm/W}_{el}]= \frac{\phi}{P_{el}}
\end{equation}

Unlike the luminous efficacy of radiation, this metric cannot be extracted from the spectrum of a light source. It is often cited in device datasheets, scientific literature and textbooks when describing the performance of light-emitting diodes. Care must be taken not to confuse this efficacy metric with efficacy of radiation, which depends only on the spectral power distribution of a light source. As it captures overall device efficacy, it depends on a large number of device properties and parameters. This makes attribution to single changes in device design or manufacturing difficult.

\subsection{Arbitrary Units}
\label{subsec:arbunit}

In spectroscopy, an arbitrary unit (abbreviated "Arb. Unit") is a relative unit of measurement to show the ratio of the intensity to a predetermined reference. In the case of Figure \ref{fig:phosphor_spectrum}, the reference is the highest point in the spectrum.

\newpage
\section{Manufacturing Cost Model}
\label{sec:costmodel}

\subsection{Structure of the Model}

\begin{figure}[h!]
    \tikzfig{./figures/costmodel}
    \centering
    \caption{Schematic diagram of the cost model showing inputs to each step and computational steps leading to cost model outputs. For description of cost variables, see definitions for equation \ref{eqn:cost_wafer}.}
    \label{fig:costmodel-schematic}
    \vspace{-8pt}
\end{figure}

The cost model adapted for this publication is a microeconomic manufacturing cost model. Within the timeframe and scope laid out in the main publication, it returns the total manufacturing cost of phosphor converted warm white light-emitting diode packages. In this computation, it considers the main economic factors associated with operating and maintaining manufacturing equipment. It does not consider costs associated with research and development or those associated with the construction of manufacturing facilities. It considers market trends through their effect on manufacturing parameters.

A schematic diagram of the cost model developed in this thesis is presented in figure \ref{fig:costmodel-schematic}. The cost model is process step based. It is split between the upstream part of the manufacturing process and the downstream part of the manufacturing process. The former is performed at the wafer level. The latter is performed at the package level. The model takes as inputs both parameters specific to manufacturing process steps (\textit{"step parameters"}) and parameters applying to all manufacturing steps (\textit{"global steps"}). The cost for each process step is then computed. The cost model returns the total and step manufacturing costs. It further considers the yield per step and returns the cumulative yield, the yielded cost per step and the yielded total manufacturing cost.

\subsection{Examples of Chip Architectures}

Figures \ref{fig:manuf_vtf_2012-1}-\ref{fig:manuf_csp_2020-2} show a simplified rendering of the manufacturing process of two differente chip architectures considered in the cost model.

    %VERTICAL THIN-FILM FLIP-CHIP

    \begin{landscape}
        \begin{figure}
            \includegraphics[width=595pt]{./figures/vtf_overview_2012-1.pdf}
            \caption{(1/2) Manufacturing process for a vertical thin-film  package flip-chip LED chip with vertical current spreading, circa 2012. Continued on next page.}
            \label{fig:manuf_vtf_2012-1}
        \end{figure}
    \end{landscape}

    \begin{landscape}
        \begin{figure}
            \includegraphics[width=595pt]{./figures/vtf_overview_2012-2.pdf}
            \caption{(2/2) Continued from previous page.}
            \label{fig:manuf_vtf_2012-2}
        \end{figure}
    \end{landscape}

    % CHIP-SCALE PACKAGE FLIP-CHIP

    \begin{landscape}
        \begin{figure}
            \includegraphics[width=595pt]{./figures/csp_overview_2020-1.pdf}
            \caption{(1/2) Manufacturing process for a chip scale package flip-chip LED chip with vertical current spreading, circa 2020. Continued on next page.}
            \label{fig:manuf_csp_2020-1}
        \end{figure}
    \end{landscape}

    \begin{landscape}
        \begin{figure}
            \includegraphics[width=595pt]{./figures/csp_overview_2020-2.pdf}
            \caption{(2/2) Continued from previous page.}
            \label{fig:manuf_csp_2020-2}
        \end{figure}
    \end{landscape}

\subsection{Computation of Manufacturing Cost}

The manufacturing process of semiconductor devices can be categorized by the level of integration at which steps are implemented. While the upstream steps in the manufacturing flow are carried out at the wafer level, the packaging steps are typically carried out at the die/package level. The total manufacturing cost per die is thus the sum of the total costs of all wafer processing steps and all die packaging steps.

\begin{equation}
\label{eqn:cost_sum}
    C \bigg[ \frac{ \$(2020) }{ \text{die} } \bigg] = P_S + C_w + C_p
\end{equation}

where

\begin{align*}
    P_S &\dots \text{sapphire substrate price} \\
    C_w &\dots \text{wafer processing cost} \\
    C_p &\dots \text{die processing cost}
\end{align*}

The total wafer processing cost and total die packaging costs are in turn the sum of all associated process steps.

\begin{align}
	C_w &= \sum_i C_i \\
	C_p &= \sum_j C_j
\end{align}

The cost of a single process step $C_i$ can now be written as

\begin{equation}
\label{eqn:cost_wafer}
    C_i \bigg[ \frac{ \$(2020) }{ \text{die} } \bigg] =\frac{1}{D}  \frac{1}{y_i}   \bigg\{ \bigg((e*p) + l + m + d +o \bigg)_i \bigg( \frac{t_i}{w_i u_i} \bigg) + \sum_{x} v_x p_x \bigg\}
\end{equation}

where the index $i$ runs over all wafer processing steps, the index $j$ runs over all die processing steps and the indices $x,y$ run over all materials.

\begin{align*}
        D &\dots \text{good die per wafer} \label{def:cost_wafer_first} \\
        y &\dots \text{process step yield} \\
        u &\dots \text{equipment utilization} \\
        p &\dots \text{power consumption} \\
        e &\dots \text{hourly electricity cost} \\
        m &\dots \text{hourly maintenance cost} \\
        d &\dots \text{hourly depreciation cost}\\
        l &\dots \text{hourly labour cost} \\
        o &\dots \text{hourly overhead cost} \\
        t_i &\dots \text{time per run} \\
        w &\dots \text{wafers per run} \\
        A &\dots \text{wafer area} \\
        v_x &\dots \text{volume of substance per wafer} \\
        p_x &\dots \text{price of substance $x$ per volume}\\
\end{align*}

The number of die per wafer $D$ depends on the total USAble wafer area. The USAble area depends on the wafer diameter, the cutting street width between the chips and the exclusion zone at the rim of the wafer.

\begin{equation}
	A_{\text{USAble}}=A_{\text{wafer}}-A_{\text{cut}}-A_{\text{exclusion}}
\end{equation}

Determining the USAble wafer area as a of these three parameters requires a numerical solution. However, following discussions in literature \cite{de2005investigation}, we approximate the number of good die per wafer as

\begin{equation}
label{eqn:dpw}
	D=\frac{\pi}{4}  \bigg ( \frac{d-2e}{\sqrt{a}+s/2} \bigg ) ^2 - \frac{\pi}{\sqrt{2}}\frac{d-2e}{(\sqrt{a}+s/2)^2}
\end{equation}

where

\begin{align*}
    d &\dots \text{wafer diameter} \\
    e &\dots \text{wafer edge exclusion zone width} \\
    a &\dots \text{die area} \\
    s &\dots \text{cutting street width} \\
\end{align*}

which gives us for the cost of a manufacturing step $C_i$ in the wafer processing category

\begin{equation}
\label{eqn:cost_wafer_full}
\begin{split}
    C_i \bigg[ \frac{ \$(2020) }{ \text{die} } \bigg] &= \bigg (  \frac{\pi}{4}  \bigg ( \frac{d-2e}{\sqrt{a}+s/2} \bigg ) ^2 - \frac{\pi}{\sqrt{2}}\frac{d-2e}{(\sqrt{a}+s/2)^2} \bigg )^{-1} \times \\
    &  \frac{1}{y_i}  \bigg\{ \bigg((e*p) + l + m + d +o \bigg)_i \bigg( \frac{t_i}{w_i u_i} \bigg) + \sum_{x} v_x p_x \bigg\}
\end{split}
\end{equation}

and the cost of a manufacturing step $C_j$ in the packaging category

\begin{equation}
\label{eqn:cost_die}
    C_j \bigg[ \frac{ \$(2020) }{ \text{die} } \bigg] = \frac{1}{y_j}  \bigg\{ \bigg((e*p) + l + m + d + o \bigg)_i  \frac{c_j}{u_j} + \sum_{x} a v_x p_x \bigg\}
\end{equation}

where $c_j$ is $\text{throughput}^{-1}$. The total cost is thus
\begin{equation}
\label{eqn:cost_total}
\begin{split}
    C= P_s &+ \sum_i \bigg \{ \frac{1}{D} \frac{1}{y_i} \bigg[ \frac{t_i}{w_i u_i} \bigg((e*p) + l + m + d +o \bigg)_i  + \sum_{x} v_x p_x \bigg] \bigg \} + \\
    & + \sum_j \bigg \{ \frac{1}{y_j} \bigg[ \frac{c_j}{u_j}  \bigg((e*p) + l + m + d + o \bigg)_i + \sum_{x} a v_x p_x \bigg ] \bigg\}
\end{split}
\end{equation}

Note that in keeping with the categorization introduced by the United States Department of energy, certain steps from these two categories are reported separately. In the wafer processing category, the epitaxy step is reported separately due to its complexity and the large share of cost carried. In the wafer processing category, the phosphor step is reported separately.

\subsection{Computation of Yielded Cost}

Devices may be damaged or otherwise rendered unUSAble during the manufacturing process. The ratio between the number of good devices per step and the number of handled devices per step is known as the yield. Optimizing this yield is critical for reducing manufacturing cost \cite{Kumar2006}. This is because cumulative yield quickly drops as the yield from manufacturing steps with below 100\% yield is multiplied. We must thus consider not only the manufacturing cost per process step, but also the cost including the yield \cite{becker2001use}\cite{becker2001using}. While there are different mathematical approaches to including yield, we follow the definition in  \cite{becker2001use}. We write for the yielded cost $C_{Y_i}$ of a step $i$

\begin{equation}
\begin{split}
\label{eqn:C_2}
    C_{Y_1} &= \frac{C_1}{Y_1} \\
    C_{Y_2} &= \frac{C_1 + C_2}{Y_1 Y_2} - C_{Y_1} = \frac{C_1 + C_2}{Y_1 Y_2} - \frac{C_1}{Y_1} = \frac{1}{Y_1 Y_2} \bigg ( C_1 (1-Y_2) +C_2 \bigg)\\
    C_{Y_i} &= \frac{ \sum_{x \leq i} C_x }{ \prod_{x \leq i} Y_x } - \frac{ \sum_{x<i} C_x }{ \prod_{x<i} Y_x }
\end{split}
\end{equation}

If a step is applied more than once, we can conveniently rewrite this in a form suited to computation within the \textit{Excel} worksheet. Assuming step $2$ is used twice, we get for the yielded cost of this step an equation of the form

\begin{align}
\label{eqn:C_2^2}
    C_{Y_2}^{(2 \times)} &= \bigg( \frac{C_1 + C_2}{Y_1 Y_2} - \frac{C_1}{Y_1} \bigg) + \bigg( \frac{C_1 + 2 C_2}{Y_1 Y_2^2} - \frac{C_1 + C_2}{Y_1 Y_2}     \bigg) \\
    &= \frac{1}{Y_1 Y_2^2} \bigg( C_1 (1-Y_2^2) +2C_2 \bigg)
\end{align}

which can be compared to equation \ref{eqn:C_2} for a more general form.

\begin{equation}
    C_2^{(n \times)} = \frac{1}{Y_1 Y_2^n} \bigg( C_1 (1-Y_2^n)+nC_2\bigg)
\end{equation}

Omitting a proof by induction, we get for the general term for a step $i>1$ repeated $n$ times an equation of the form

\begin{equation}
\label{eqn:yielded_cost}
    C_{Y_{i>1}}^{(n \times)} = \frac{1}{Y_i^{n-1} \prod_{x \leq i} Y_x} \bigg( nC_i + \sum_{x < i} C_x (1-Y_i^n) \bigg)
\end{equation}

This cummulative approach to yielded cost is different from the approach taken in the original \textit{LEDCOM} model. It uses what in literature is described as an \textit{"itemized approach"} to yielded cost \cite{becker2001use}. The yielded cost of a single process step $f$ is described by

\begin{equation}
	f_\text{single} = \frac{i+s}{y}
\end{equation}

where

\begin{align*}
    i &\dots \text{material cost of previous step} \\
    s &\dots \text{step cost}
\end{align*}

For process steps that are performed more than once, a series expression is used

\begin{align}
\label{eqn:series}
    f_{\times 2} &=  \frac{\frac{i+s}{y}+s}{y} \\
    f_{\times n} &= \frac{i + s(1+y+y^2+ \dots + y^{n-1})}{y^n}
\end{align}

This itemized approach serves as a convenient approximation, but its cumulative contributions do not equal the total yielded cost of the entire manufacturing process

\begin{equation}
    f_\text{total} = \frac{\sum_i^n s_i}{\prod_i^ny_i} \neq \sum_i^n f_i
\end{equation}

where $n$ is the total number of steps. This is due to the approximation introduced through the series approximation in equation \ref{eqn:series}.

\subsection{Computation of the Contribution of Variables}

To quantify the drivers of cost reductions across all process steps, one would need to identify the magnitude of contribution to cost reductions made by single variables in equations \ref{eqn:cost_wafer} and \ref{eqn:cost_die}. Mathematically, given a function $C$ which describes the cost associated with manufacturing a single unit at time $t$,

\begin{equation}
C=ab+cd
\end{equation}

and manufacturing variables $a,b,c,d$, we are looking for the contribution $\Delta C_{a}$ made by a single variable $a$ to the total change in the cost function $\Delta C$ between points $t_0,t_1$ such, that

\begin{equation}
\Delta C = C(t_1)-C(t_0) = \sum_{i=a, \dots, d} \Delta C_i
\end{equation}

The infinitesimal contribution to the total cost by the infinitesimal change in a cost model variable is defined through the total differential of the cost function \ref{eqn:cost_sum} as

\begin{equation}
\text{d}C(x_1 (t), x_2(t), \dots) = \sum_i \frac{\partial C }{\partial x_i}     \frac{\text{d}x_i}{\text{d}t} = \sum_i \frac{\partial C }{\partial x_i}  \Delta x_i
\end{equation}

where $x_i$ is an arbitrary cost variable. The contribution of the change $\Delta x_1$ in variable $x_1$ to total cost $C$ over the period $t_0 < t < t_1 $ is then

\begin{equation}
\Delta C_{x_1} = \int_{t=t_1}^{t_2} \frac{\partial C }{\partial x_1} \frac{\text{d}x_1}{\text{d}t} \text{d}t
\label{eqn:integral_1}
\end{equation}

However, data on the cost model variables is not available in continuous time. Disaggregating the contribution of single variables to the total cost reduction is thus not straightforward in our model. This problem does not arise in cost models which compute cost changes directly, such as \cite{nemet2012solar} \cite{goodrich2013assessing}. The following discussion follows an approach developed by Kavlak et al. \cite{kavlak2018evaluating}. For the detailed derivation, we refer to this publication.

The cost function $C$ as a function of a vector of cost model variables $\vec{r}=(r_1,r_2,\dots)$ is defined as

\begin{equation}
C(\vec{r}) = C(r_1,r_2, \dots) = \sum_i C_i
\end{equation}

where

\begin{equation}
C_i(\vec{r}) = C_i^0 \prod_w g_{iw}(r_w)
\end{equation}

Using logarithmic differentiation, the integral from equation \ref{eqn:integral_1} can be rewritten as

\begin{equation}
\Delta C_x = \int_{t=t_0}^{t_1} C(t) \frac{ \partial \ln C }{ \partial x } \frac{ \text{d} x }{ \text{d} t} \text{d} t
\end{equation}

where for $C(t)$ a constant $C(t) \approx \tilde{C} $ can be chosen such that $\Delta C_{x_i} = \Delta C$. In practice, this constant value can be approximated as $\tilde{C} \approx \frac{2}{3} C_i^\text{geo} + \frac{1}{3} \overline{C_i}$. The contribution of a single cost model variable $r_z$ can then be written as

\begin{equation}
\Delta C_z (t_1,t_2) \approx \sum_i \tilde{C_i} \ln \frac{g_{iz}(t_2)}{g_{iz}(t_1)}
\end{equation}

This approach was used to determine the effect of spillovers on device performance, displayed in figure \ref{fig:breakthroughs_efficiency}. Due to time constraints, this approach could not be used to determine the effect of spillovers on manufacturing cost. Future work to quantify the impact of different cost drivers, including breakthroughs and changes in the manufacturing variables of equation \ref{eqn:cost_sum}, could this method.

\subsection{Examples of Input Data Considered}

\subsubsection{Sapphire Wafers}

Sapphire wafers form the substrate on which all other layers of the light-emitting diode wafer are grown. Being transparent to radiation in the visible spectrum, it is not removed after growth in the Classical architecture or the Thin-Film Flip-Chip architecture. In the Vertical Thin-Film architecture, it is removed by means of a laser-lift-off process. Wafers can be either unpatterned or patterned, where the latter has become commonplace in 2020 due to the beneficial properties that microstructures on the surface have on layer growth \cite{wuu2009defect} and light-extraction efficiency \cite{lee2006enhancing}. The price of sapphire substrates has decreased significantly since the year 2000, as shown in figure \ref{fig:sapphire_prices}. This can be attributed not only to the lighting industry, but more importantly to increased demand from electronics manufacturing \cite{yole2015sapphire}, where sapphire glass is used to protect screens and sensor interfaces from scratches \cite{khattak2016world}. Wafer sizes used in manufacturing have also increased, due to the favourable economics of large wafer processing. The market is dominated by U.S. based \textit{Rubicon Technology} and Russian based \textit{Monocrystal}. The prevalence of wafer diameters used is shown in figure \ref{fig:wafer_size}.

\begin{figure}[h!]
    \includegraphics[width=14.5cm]{./figures/wafer_size.pdf}
    \includegraphics[width=15cm]{./figures/sapphire_prices.pdf}
	\caption{Top: Historical data for sapphire substrate prices of different surface properties and diameters. Shown is data for polished surface (NPS) and patterned substrates (PSS). \textit{Monocrystal} denotes the Russian manufacturer of the same name. Dashed lines are projections from the previous year. Sources: \cite{monocrystal2020private}\cite{yole2011sapphire}\cite{yole2015sapphire}. Bottom: Prevalence of sapphire wafer size used in the manufacturing of light-emitting diodes. Hatched bars are projections given by the sources. Sources: \cite{veeco2013}\cite{Scholand2012}\cite{yole2015sapphire}}
	\label{fig:wafer_size}
\end{figure}

\subsection{Preliminary Sensitivity Analysis}

\begin{figure}[h]
	\centering
    \includegraphics[width=\textwidth]{2_SSL_EES/article/figures/costmodel_sensitivity.pdf}
	\caption{Sensitivity analysis of selected cost model parameters. Note the decreasing sensitivity to cost model parameters as the number of die per wafer increases. Compare table \ref{tab:sensitivity}. Abbreviations: resp. - respective}
	\label{fig:sensitivity}
\end{figure}

A sensitivity analysis of the cost model has been performed using parameter variations listed in table \ref{tab:sensitivity}. The amount of parameter variation was chosen to encompass the identified range of the value of the parameter in different manufacturing setups of each considered year. As an example, the lower range in the variation for metal cost (+50\%,-0\%) was chosen because the metal prices in the model are price quotes from the United State Geological Survey price database. Industry metals are not sold below the price of the raw material and often the markup is small compared to the price of the material. The upper range was chosen, because a survey of industrial metal suppliers for semiconductor manufacturing showed that the largest markup was below 50\%. The results of the analysis are shown in figure \ref{fig:sensitivity}. The cost model is generally more sensitive to variation of parameters at smaller wafer diameters. The most sensitive parameters are global parameters, such as yield or average equipment throughput.

\begin{table}[H]
\small
    \begin{tabularx}{\textwidth}{ |X|l|l|l|l|l|l|l|X|}
        \hline
            \textit{Parameter} & \textit{Unit} & \textit{2003} & $\pm [\%]$ & \textit{2012} & $\pm [\%]$ & \textit{2020} & $\pm [\%]$ & \textit{Source} \\
        \hline
            Cleanroom Cost & \$/m$^2$ & 3000 & +16,-16 & 3000 & +16,-16 & 3000 & +16,-16 & Figure \ref{fig:electricity+cleanroom_prices} \\
        \hline
            Manhours & FTE & 100\% & +50,-50 & 100\% & +50,-50 & 100\% & +50,-50 & I \\
        \hline
            Equip. Discount & \% of \$ & 0\% & +25,-25 & 0\% & +25,-25 & 0\% & +25,-25 & I, \cite{Appleyard2001} \\
        \hline
            Overall Yield & \% & 100\% & +25,-25 & 100\% & +25,-15 & 100\% & +25,-25 & I, \cite{lumi2012yield}\cite{ledsmag2012} \newline \cite{systemplus2015reverse}\cite{ledcomv2} \\
        \hline
            Inspec. Yield Savings & \%/inspec. & 0.5\% & +80,-80 & 0.5\% & +80,-80 & 0.5\% & +80,-80 & \cite{mckinseyyield} \\
        \hline
            Overall Throughput & UPH or h$^{-1}$ & 100\% & +50,-50 & 100\% & +50,-50 & 100\% & +50,-50 & Datasheets \\
        \hline
            Wafer Diameter & mm & 100 & +0,-50 & 150 & +33.3,-33.3 & 200 & +0,-25 & I, Figure \ref{fig:wafer_size} \\
        \hline
            Edge Exclusion & mm & 7 & +0,-50 & 5 & +40,-0 & 5 & +40,-20 & \cite{ledsmagexclusion}\cite{rubiconexclusion} \newline \cite{xiamenexclusion}\cite{american2007annual} \\
        \hline
            Cutting Width & $\mu$m & 100 & +50,-25 & 75 & +33.3,-20 & 20 & +300,-50 & \cite{masaki2000division}\cite{ils2005width} \newline \cite{photonics2010width}\cite{discowidth} \\
        \hline
            Metal Prices & \$/kg & 100\% & +50,-0 & 100\% & +50,-0 & 100\% & +50,-0 & Datasheets \\
        \hline
            Electricity Price & \$/kWh & 100\% & +50,-50 & 100\% & +50,-50 & 100\% & +50,-50 & \cite{eia2000electric}\cite{eia2019electric} \\
        \hline
            Saph. Subst. Price & \$ & 40 & +12.5,-12.5 & 10 & +100,-20 & 3 & +66.6,-33.3 & Figure \ref{fig:sapphire_prices} \\
        \hline
            Phosphor Prices & \$/g & 100\% & +50,-50 & 150 & +0,-0 & 200 & +0,-0 & I, \cite{yole_phosphor_2012}\cite{yole2017phosphor} \\
        \hline
        \end{tabularx}
    \caption{Cost model sensitivity analysis parameter list. For the sensitivity analysis performed using these variables, see figure \ref{fig:sensitivity}. Units for values in columns \textit{2002}-\textit{2020} are indicated in column \textit{Units}. If values in columns \textit{2003}-\textit{2020} are instead are given in \%, this indicates that the parameters were varied by a set percentage from their respective model baselines.}
    \label{tab:sensitivity}
\end{table}

\subsection{Comparison with Reported Industry Data}

We note differences between our model and costs reported to the US Department of Energy (DoE) by industry. For instance, the share of the epitaxy step is larger in the DoE model. This can in part be explained by our model relying on state-of-the-art equipment, while industry might not run low-power and mid-power chip production on these, more expensive, reactors. The packaging part of the manufacturing process is increasing in importance, which has been confirmed by researchers and industry reports on wafer-level packaging.

\begin{figure}[h]
	\centering
    \includegraphics[width=\textwidth]{2_SSL_EES/article/figures/costmodel_calibration.pdf}
	\caption{Modelled manufacturing cost for light-emitting diode packages split by manufacturing step category. The top graph shows the data from LEDCOM model and round table discussions published by the United States Department of Energy (DoE). The bottom graph shows the results of the cost model developed in this thesis. Hatched bars are projections given by the sources. Inset plot shows the number of die per wafer (DPW) calculated for different wafer sizes used in the model. Sources for top panel: \cite{doe2010solid}\cite{doe2011solid}\cite{doe2012solid}\cite{doe2013solid}\cite{doe2014solid}\cite{doe2015solid}\cite{doe2016solid}. Sources for bottom panel: own elaboration based on the cost model described in Section \ref{sec:costmodel}. Sources for Inset Plot: own elaborations based on numerical approximations for die-per-wafer estimations provided in \cite{de2005investigation}, informed by expert interviews.}
	\label{fig:costmodel_calibration}
\end{figure}

Figure \ref{fig:costmodel_calibration} shows a comparison of the relative manufacturing cost by manufacturing step categories between the cost model and data released by the United States Department of Energy (DoE). Note that any comparison between the figures must consider: Our model assumes state-of-the-art equipment, while the DoE industry round table reports data from factories which use older equipment, as well as newer equipment. Our model presently considers only low-power and mid-power packages, while the DoE reports consider high-power packages also. Our model assumes a virtual North American manufacturing location, while the majority of manufacturers produce in Asia. We note that the share of the substrate price in the DoE data is much larger than in our model. This can in part be explained by figure \ref{fig:sapphire_prices}, which already shows how projections overestimated the actual price for 2015-2020. We note also that the share of the epitaxy step is larger in the DoE model. This can in part be explained by our model relying on state-of-the-art equipment, while industry might not run low-power and mid-power chip production on these, more expensive, reactors. In conclusion, our model fits industry forecasts due to the way it is set up. The packaging part of the manufacturing process is increasing in importance, which has been confirmed by researchers and industry reports on WLP\cite{Lee2011WPL}\cite{Xie2013}\cite{ledsmag2017WLP}.

\newpage
\section{Phosphors}

\subsection{History of Identifies Phosphor Inventions}

\subsubsection{YAG and YGAG Phosphors}

Prior to Nakamura’s invention of highly efficient blue LEDs in the 1990s, Japan’s Nichia Corporation had not sold commercially successful semiconductor products, instead specializing in phosphors for cathode ray tubes (CRT) and fluorescent lamps \cite{nakamura2013blue}  . Nevertheless, extensive firm expertise in this area helped Nichia’s Yoshinori Shimizu formulate the principles of using CRT phosphors to convert blue light from Nakamura’s LEDs into white light in 1994 \cite{shimizu1994sheet}\cite{cho2017white}. By 1996, Shimizu and his colleagues developed \cite{bando1996}\cite{shimizu1999light} the first practical LED application of a well-known Yttrium Aluminium Garnet (YAG) CRT phosphor activated with cerium \cite{blasse1967new}, enabling the first commercial white LED products manufactured and sold by Nichia since late 1996 \cite{bando1998development}\cite{cho2017white}. 

Importantly, the YAG phosphor does not exhibit the spectral properties desirable in general illumination applications (see Figure 8 in the main text). An early solution to this problem, which was first discovered in the late 1967 \cite{holloway1969optical} and suggested for LEDs by the same team at Nichia in 1996 \cite{bando1998development}\cite{shimizu1999light}, was to use the gadolinium-doped red-shifted YAG phosphor (YGAG). Used in combination with red-emitting sulfide phosphors, by 2002 it helped bring to the market the first generation of warm white light LED products, e.g., those produced by Lumileds \cite{Mueller2002}. However, sulfide phosphors led to accelerated deterioration of sensitive parts of LED devices and became less efficient as operating temperatures increased. New chemically stable and non-toxic red phosphors were needed. 

\subsubsection{258, SLA and SALON phosphors}

In 1997, Hubert Huppertz and Wolfgang Schnick, working at the University of Bayreuth in Germany, synthesized the first compound in a new class of rare earth nitridosilicate materials \cite{Huppertz1997} later dubbed “258” due to a proportion of elements in its chemical formula. The luminescent properties of these materials were identified by the Schnick’s group, by then at Ludwig-Maximilians University of Munich, in 2000 \cite{Hppe2000} after a suggestion made to Schnick at a conference following earlier reports of good luminosity properties of europium-doped compounds \cite{Qiua1998}. U.S.-based LED manufacturer Lumileds applied for a patent for the first class of red LED phosphors based on the 258 nitridosilicate chemistry in 2002 \cite{mueller2004phosphor}. The first use of the 258 phosphor in a commercial “Luxeon” LED package was then reported in a joint publication co-authored by inventors from Lumileds and researchers from the Schnick’s group in 2005 \cite{MuellerMach2005}.

Further efforts in LED phosphor development were directed towards synthesizing a red narrow-band phosphor. Narrow LED emission peak widths yield the highest luminous efficacy of radiation, as in this case less light is emitted in the far-red range of the spectrum in which the human eye is not very sensitive. After synthesizing several narrow-band phosphors emitting in yellow \cite{Hppe2004} and cyan \cite{Kechele2009}, the Schnick’s group identified the local cubic cation coordination structure of the cyan phosphor compound as the reason for its narrow band width \cite{lumi2016narrow}. A search for a structurally analogous nitride compound with the narrow red instead of the cyan emission was undertaken. After several unsuccessful attempts, the sought-after cuboidal nitride compound was found in a 2008 publication led by Francis DiSalvo \cite{Park2008Sr}. Based on information provided in this work, Schnick and colleagues synthesized and studied the spectral properties of a new narrow band red SLA phosphor in 2013 \cite{schmidt2013new}\cite{Pust2014}\cite{schmidt2017phosphors}. The material was introduced in commercial LED devices by Lumileds in 2015 \cite{lumi2016narrow_whitepaper}. 

The most recent red narrow-band phosphor breakthrough included in Table 3 and Figure 8 in the main text, indicated as SALON, has been under development during the late 2010s by a group of Austrian and German researchers that included Huppertz, the discoverer of the “258” material, working in collaboration with Osram, another major LED manufacturer \cite{seibald2019phosphor}\cite{Hoerder2019}\cite{Hoerder2020}. The first patent application for this phosphor was filed in 2016. The SALON phosphor is a derivate of the SLA phosphor. Therefore, it is the only innovation related to consumer experience metrics identified in our study that seemingly not involved technology spillovers.

\subsubsection{PFS Phosphor}

Down conversion with ultra-narrow-band phosphor can achieve the highest spectral efficiency. However, few such phosphors have been identified, with even less exhibiting desirable material properties such as thermal stability \cite{Phillips2007}. The first commercially successful ultra-narrow-band red phosphor was developed by General Electric (GE). It is based on a potassium fluorosilicate (PFS) compound activated with manganese ions. Its luminescence was first recorded by Paulusz at GE in 1972 \cite{paulusz1973efficient}. In the early 2000s, while searching for potential new LED phosphor materials for GE’s lighting business at GE Lumination, Emil Radkov rediscovered Paulusz’s findings in the literature. Following extensive research on PFS chemical synthesis and material properties conducted in collaboration with the University of Sofia in Bulgaria, Radkov’s Alma Mater, the PFS phosphor had been under development at GE since 2005 \cite{radkov2006red}\cite{radkov2009red}. This work, supported by public funding from the U.S. Department of Energy (DOE) Solid-State Lighting program , resulted in a series of critical improvements in the PFS phosphor properties \cite{Setlur2010}\cite{lyons2012color}, eventually enabling its commercialization under the “TriGain” brand in 2015 \cite{trigain_spectrum}\cite{setlur2015trigain}\cite{Murphy2015}.

\subsubsection{Quantum Dots for Light Down-Conversion}

Quantum dots (QD) are semiconductor nanocrystals whose quantum size effects make QDs behave as “artificial atoms”. Semiconductor quantum dots were first synthesized in the Soviet Union in 1981 \cite{ekimov1981quantum} and at Bell Labs in the U.S. in 1983 \cite{Rossetti1983}. Luminescent properties of quantum dots were first empirically observed in 1984 \cite{fojtik1984photo} and extensively studied in the early 1990s. The key feature of QD luminescence discovered in those studies is that its colour is determined by the QD particle size, making it possible to create pure monochromatic blue, green and red light sources just by tuning the QD size. The first application of QDs in LEDs was reported in 1994 in an electroluminescent hybrid QD-polymer LED. However, this LED type could not be used in general illumination due to its very low luminous efficacy. An alternative application of QDs as a kind of a “phosphor” for light down conversion from an LED light source was proposed in the early 2000s as part of the U.S. Department of Energy (DOE)-funded “A Revolution in Lighting“ project at Sandia National Laboratory \cite{simmonsfinal}. This concept was successfully demonstrated by Sandia researchers on a commercial LED in 2003 \cite{shea_rohwer_development_2004}\cite{noauthor_sandia_nodate} and was swiftly taken up and advanced further by a group in Taiwan \cite{chen_white-light_2005}\cite{hsueh-shih_chen_ingan-cdse-znse_2006} The first commercial application of QDs in an LED lamp was brought about by a collaboration between an MIT-born startup QD Vision and the U.S.-based luminaire manufacturer Nexxus Lighting in 2009 \cite{ledprof_nexxusqd}, \cite{bourzac2013quantum}. However, rapid advances in the spectral and conversion performance of down-conversion phosphors and high manufacturing cost of quantum dots resulted in the discontinuation of this product. After finding market success in display backlighting first demonstrated by Samsung in 2010 \cite{Jang2010} and commercialized by QD Vision in Sony television sets in 2013 \cite{bourzac2013quantum}, QDs returned to the general lighting market in products offered by Lumileds \cite{noauthor_global_2017}\cite{noauthor_quantum_2020} around 2017 and Osram in 2019 \cite{osramqdots} in the form of mid-power LED packages that combined QDs with traditional phosphors for light down conversion.

\subsection{Spectral Data of Identified LED Phosphor Innovations}

\begin{figure}[H]
	\centering
    \includegraphics[width=0.95\textwidth]{2_SSL_EES/article/figures/phosphor_spectrum-comparison.pdf}
	\caption{Two key metrics associated with red phosphor innovations in light-emitting diodes and their associated spectral data. Top panel: Luminous efficacy of radiation (LER) and colour rendering index (CRI) of white LED devices represented in Main Article Figure 7 (Historical Improvements in Consumer Experience Metrics of Phosphor-Converted White LED). Years indicate the earliest identified representative white LED products with published spectral data that used phosphor innovations listed in Main Article Table 2. The desirable direction of improvements towards higher luminous efficacy at higher CRI is indicated by a red arrow. Metrics were calculated from spectral data shown in the two bottom rows using the \texttt{colour-science} package for Python \cite{colour-science_software}. Bottom two rows of panels: Corresponding spectral data. The luminosity function, describing the wavelength-dependent sensitivity of the human eye, is shown for reference in red in each panel. Note that peaks or large tails of the device emission spectrum at the far ends of the luminosity function are not desirable. This is because photons of the corresponding energy are lost to the human observer and count towards the spectral loss channel. Abbreviations: Arb. Units - Arbitrary Units (compare section \ref{subsec:arbunit}). Plot legends indicate the years of publication of the spectral data and phosphor mixtures used with the following designations: ? -  other parts of phosphor mixture not disclosed, $^*$ - \ch{CaSrS:Eu^{2+}}, $^\star$ - $\beta-$SiAlON:Eu$^{2+}$, $^\dagger$ - \ch{Lu_3Al_5O_{12}:Ce^{3+}}, $^\ddagger$ - \ch{(Ba,Sr)_2Si_5N_8:Eu^{2+}}. Full chemical formulas for displayed phosphors: \ch{Sr[Li2Al2O2N2]:Eu^{2+}}(SALON), \ch{Sr[LiAl_3N_4]:Eu^{2+}} (SLA), \ch{K2SiF6:Mn^{4+}} (PFS), \ch{Sr2Si5N8:Eu^{2+}} (258), Quantum Dots (QD). Source (top panel): own elaboration based on spectral data. Source (spectral data): adapted from published spectral data for LEDs with the following phosphors: YAG \cite{bando1998development}, YGAG \cite{Mueller2002}, 258 \cite{MuellerMach2005}, SLA \cite{Pust2014}, PFS \cite{trigain_spectrum}, QD \cite{lumileds2016qd}\cite{osram2019qd}, SALON \cite{Hoerder2019}.}
\label{fig:phosphor_spectrum}
\end{figure}

\section{Anonymized List of Interviewees}

\begin{table}[h!]
\small
    \centering
    \begin{tabular}{|l|l|l|l|l|}
    \hline
        \textbf{\#} & \textbf{sector} & \textbf{role} & \textbf{country} & \textbf{expertise} \\ \hline
        1 & academia & sr. researcher & UK & epitaxy \\ \hline
        2 & industry & consultant, form. sr. researcher & USA & device architecture \\ \hline
        3 & industry & consultant, form. head of R\&D & Germany & epitaxy \\ \hline
        4 & academia & professor & Austria & phosphors \\ \hline
        5 & industry & consultant, form. head of R\&D & USA & device architecture \\ \hline
        6 & consulting & consultant, form. sr. technical advisor & USA & device architecture \\ \hline
        7 & academia & professor & Germany & phosphors \\ \hline
        8 & government & R\&D manager & USA & device architecture \\ \hline
        9 & consulting & consultant & USA & device applications \\ \hline
        10 & academia & professor & France & device physics \\ \hline
        11 & industry & sr scientist, form. head of R\&D & USA & device architecture \\ \hline
        12 & industry & principal scientist & Germany & phosphors \\ \hline
        13 & industry & form. head of R\&D & USA & phosphors \\ \hline
    \end{tabular}
    \label{tab:interviews}
    \caption{List of domain experts with whom semi-structured interviews were conducted. \\ Abbreviations: form. - former, sr. - senior}
\end{table}

\clearpage
\newpage
\section{Complete List of Sources for Figures}
\label{sec:sources}

\subsection{Figure 1 (Historical Development of Luminous Efficacy)}

The complete list of sources is organized by the different technologies described in the figure:

\begin{table}[h!]
    \begin{tabularx}{\textwidth}{|l|X|}
    \hline
    \textbf{Technology} & \textbf{References} \\
    \hline
    LED & own research, compare \cite{zenodo_weinold_led_history} \\
    \hline
    CFL (<1984) & \cite{Bouwknegt1982}\cite{Vrenken1983} \\
    \hline
    CFL (1984-2011) & \cite{eger2018origin} \\
    \hline
    CFL (>2011) & \cite{Guan2015} \\
    \hline
    Fire, Incandescent, HID & \cite{azevedo2009transition} augmented by own calculations based on \cite{benesch1905beleuchtungswesen} \\
    \hline
    Max. efficacy & \cite{Murphy2012} (in contrast to the original value given in \cite{azevedo2009transition}) \\
    \hline
    \end{tabularx}
\end{table}

\subsection{Figure 2 (Historical Development of Lamp Prices)}

The complete list of sources is organized by different sources:

\begin{table}[h!]
    \begin{tabularx}{\textwidth}{|l|X|}
    \hline
    \textbf{Source of References} & \textbf{References} \\
    \hline
    Stiftung Warentest (Germany) & \cite{Warentest2008}\cite{Warentest2009_1}\cite{Warentest2009_2}\cite{Warentest2010_1}\cite{Warentest2010_2}\cite{Warentest2011}\cite{Warentest2012}\cite{Warentest2013}\cite{Warentest2014_1}\cite{Warentest2014_2}\cite{Warentest2015}\cite{Warentest2016_1}\cite{Warentest2016_2}\cite{Warentest2018} \\
    \hline
    Konsument (Austria) & \cite{Konsument2010} \\
    \hline
    Which (UK) & \cite{Which2020} \\
    \hline
    Industry Periodicals & \cite{PM2020} \\
    \hline
    Government Reports & \cite{council2013assessment} \\
    \hline
    \end{tabularx}
\end{table}

\subsection{Figure 3 (Historical Evolution of LED Chip Architectures)}

The complete list of sources is organized by different sources:

\begin{table}[h!]
    \begin{tabularx}{\textwidth}{|l|X|}
    \hline
    \textbf{Source of References} & \textbf{References} \\
    \hline
    Scientific Publications & \cite{plossl2010wafer}\cite{bierhuizen2007performance}\cite{gencc2019distributed}\cite{chong2014performance} \\
    \hline
    Patents & \href{https://worldwide.espacenet.com/patent/search?q=pn\%3DJPH11168235A}{JPH11168235A}, 
\href{https://worldwide.espacenet.com/patent/search?q=pn\%3DDE19921987A1}{DE19921987A1}, 
\href{https://worldwide.espacenet.com/patent/search?q=pn\%3DJPH11340514A}{JPH11340514A}, 
\href{https://worldwide.espacenet.com/patent/search?q=pn\%3DJP2001044498A}{JP2001044498A}, 
\href{https://worldwide.espacenet.com/patent/search?q=pn\%3DUS2005045893A1}{US2005045893A1}, 
\href{https://worldwide.espacenet.com/patent/search?q=pn\%3DUS2002093023A1}{US2002093023A1}, 
\href{https://worldwide.espacenet.com/patent/search?q=pn\%3DUS2006273339A1}{US2006273339A1}  \\
    \hline
    Other Publications & \href{https://web.archive.org/web/20170801160530/https://www.energy.gov/sites/prod/files/2015/02/f19/craford_innovation_sanfrancisco2015.pdf}{2015 presentation (www.energy.gov)} \newline
\href{https://web.archive.org/web/20170715230721/https://www.energy.gov/sites/prod/files/2016/02/f29/sun_china_raleigh2016.pdf}{2016 presentation (www.energy.gov)} \newline
\href{http://web.archive.org/web/20160425025936/https://www.slideshare.net/Yole_Developpement/yole-led-packagingjanuary2013reportsample}{2013 presentation (www.slideshare.net)} \\
    \hline
    \end{tabularx}
\end{table}

\subsection{Figure 4 (Historical Developments in Device Sub-Efficiencies)}

The complete list of sources is organized by the sub-efficiencies (panels) described in the figure:

\begin{table}[h!]
    \begin{tabularx}{\textwidth}{|l|X|}
    \hline
    \textbf{Technology} & \textbf{References} \\
    \hline
    Panel A1 ($V_f$) & \cite{nichia2001data}\cite{lumi2002data}\cite{gen2005data}\cite{candlepwr2005data}\cite{lumi2006data}\cite{lumi2007data}\cite{nichia2008data}\cite{lumi2008data}\cite{osram2008data}\cite{jeong2011high}\cite{osram2012data}\cite{osram2013data}\cite{osram2014data} \newline \cite{lumi2016data_1}\cite{lumi2016data_2}\cite{epistar2017data}\cite{osram2017data_1}\cite{osram2017data_2}\cite{samsung2017data}\cite{samsung2018data}\cite{osram2018data}\cite{epistar2018data}\cite{lumi2019data} \\
    \hline
    Panel A2 (Droop) & Data calculated from luminous intensity curves of respective device datasheets: \cite{datasheet_osram_topled}\cite{osram2008data}\cite{osram2008gdplus}\cite{osram2018csp}\cite{datasheet_lumileds_lux1}\cite{lumi2008data}\cite{lumi2016data_1}\cite{lumi2016data_2}\cite{samsung2018data} \\
    \hline
    Panel B1 (IQE) & own research, compare \cite{zenodo_weinold_led_history} \newline
\cite{doe_ssl_multiyear_2006}\cite{doe_ssl_multiyear_2007}\cite{doe_ssl_multiyear_2008}\cite{doe_ssl_multiyear_2009}\cite{doe_ssl_multiyear_2010}\cite{doe_ssl_multiyear_2011}\cite{doe_ssl_multiyear_2012}\cite{doe_ssl_multiyear_2013}\cite{doe_ssl_multiyear_2014}\cite{doe_ssl_rnd_2015}\cite{doe_ssl_rnd_2016} \\
    \hline
    Panel B2 (LEE) & \cite{lee2005analysis}\cite{krames2007status}\cite{Jang2004}\cite{Horng2013}\cite{Liao2010}\cite{HungWenHuang2005}\cite{Leem2007}\cite{Huang2008}\cite{Wang2009}\cite{Huh2003}\cite{Horng2008}\cite{Gao2008}\cite{Chang2003}\cite{Zhou2012} \newline \cite{ChunJuTun2006}\cite{Hua2009}\cite{Matioli2010}
\cite{lee2005analysis}\cite{Zhu2015}\cite{Ding2015}\cite{Taki2019}\cite{Shchekin2006}\cite{Hu2016}\cite{Horng2010}\cite{Lin2016}\cite{Yue2018}\cite{Zhao2012}\cite{Zhu2015}\newline \cite{Ding2015}\cite{wierer2001high}\cite{Steigerwald2002}\cite{DaeSeobHan2006}\cite{Wang2006}\cite{Lee2007}\cite{Shen2007}\cite{Huang2006}\cite{Zhmakin2011} \\
    \hline
    \end{tabularx}
\end{table}

\newpage
\bibliographystyle{IEEEtran}
\bibliography{bibliography}

\end{document}
