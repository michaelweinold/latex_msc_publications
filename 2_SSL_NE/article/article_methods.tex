\documentclass[parskip=full]{article}

% functionality
% formatting
\usepackage[utf8]{inputenc} % allow utf-8 input
\usepackage[T1]{fontenc} % use 8-bit T1 fonts  (allows for direct use of ö,ü,etc.)

% math typesetting
\usepackage{amsmath}
\usepackage{amssymb}
\usepackage{amsfonts}

% maths definitions, theorems, etc.
\usepackage{amsthm}

% color
\usepackage{color}
\usepackage{xcolor}

% layout
\usepackage{layout}
\usepackage{lipsum}

% cross-referencing and hyperlinks
\usepackage{hyperref}
\usepackage{url}
\usepackage{doi}

% figures
\usepackage{graphicx}
\usepackage{subfig}
\usepackage{wrapfig}

% tables
\usepackage{booktabs}
\usepackage{multirow}
\usepackage{caption} 
\usepackage{float}

% enumeration
\usepackage{enumitem}

% embedding pages
\usepackage{pdfpages}

% multi-line comments
\usepackage{comment}

% landscape orientation
\usepackage{rotating}
\usepackage{pdflscape}

% Gantt charts
\usepackage{pgfgantt}

% footnotes
\usepackage{footnote}

% code
\usepackage{listings}

% matrices and tables
\usepackage{nicematrix}
\usepackage{varwidth}
% \usepackage{tabularx} do not load package tabularx, instead use package nicematrix

% document structure
\setcounter{secnumdepth}{5} % enable numbered sub-sub-sections etc.

% custom header size
\usepackage{titlesec}

% custom titles
\usepackage{titling}

% customized references (make "Figure 1" a link, not just "1")
\usepackage[capitalise, nameinlink]{cleveref}

% customized frames around text etc.
\usepackage{mdframed}

% tikz
\usepackage{tikz}
\usetikzlibrary{fit}
\usetikzlibrary{calc,shadings,patterns}

% calligraphy
\usepackage{calligra}

% chemical formulas
\usepackage{chemformula}

% highlighting
\usepackage{soul} % to be used together with xcolor

% layout
% column layout
\usepackage{multicol}
\setlength{\columnsep}{0.75cm}

% paragraphs
\usepackage[skip=0.5\baselineskip]{parskip}

% geometry
\usepackage[
    margin = 3cm,
    top = 3cm,
    bottom = 3cm
]{geometry}

% header size
\titleformat*{\section}{\large\bfseries}%{\thesection.}{\hspace{0cm}}{}
\titleformat*{\subsection}{\normalsize\bfseries}%{\thesection.}{\hspace{0cm}}{}
\titleformat*{\subsubsection}{\normalsize\bfseries}
\titleformat*{\paragraph}{\normalsize\bfseries}
\titleformat*{\subparagraph}{\normalsize\bfseries}

% custom headers
\usepackage{fancyhdr}

% bibliography
\usepackage[
    backend=biber,
    style=ieee
]{biblatex}

% show DOI URL (https://doi.org/XXX.XXXXX.XXXX), instead of publisher URL (https://springer.com/XXXX)
% cf. https://tex.stackexchange.com/a/616241
\DeclareSourcemap{
  \maps[datatype = bibtex]{
    \map{
      \step[notfield = keywords, final]
      \step[fieldsource = doi, final]
      \step[fieldset = url, null]
    }
    \map{
      \step[fieldsource = keywords, notmatch = \regexp{\bprimary\b}, final]
      \step[fieldsource = doi, final]
      \step[fieldset = url, null]
    }
  }
}
\AtEveryBibitem{
    \clearfield{urlyear}
    \clearfield{urlmonth}
}
\addbibresource{bibliography.bib}

%% METADATA %%%%%%%%%%%%%%%%%%%%%%%%%%%%%%%%%%%%%%%%%%%%%%%%

\hypersetup{
    pdftitle={Methods},
    pdfauthor={Michael Weinold, Sergey Kolesnikov, Laura Diaz Anadon},
}

%% MAIN DOCUMENT %%%%%%%%%%%%%%%%%%%%%%%%%%%%%%%%%%%%%%%%%%%

\begin{document}

\setlength{\fboxsep}{10pt}
\fbox{
    \parbox{\textwidth}{
        \textit{\textsc{Methods Information}} for: \\
        \newline 
        M. Weinold$^{1,2,3}$, S. Kolesnikov$^3$, L.D. Anadon$^{3,4}$ \\
        "Rapid technological progress in white light-emitting diodes and its sources in innovation and technology spillovers" \textit{Nature Energy} (2023) \\
        \newline
        $^1$ Technology Assessment Group, Paul Scherrer Institute, Switzerland \\
        $^2$ ETH Zurich, Zurich, Switzerland \\
        $^3$ CEENRG, Dept. of Land Economy, University of Cambridge, UK \\
        $^4$ Belfer Center for Science and International Affairs, Harvard University, Cambridge MA, USA
    }
}

\section{Overview}
\label{sec:methods}

The evolution of LED device architecture and performance as well as the progress in understanding the underlying physical phenomena are well covered in the scholarly literature and patents. However, information provided in such sources is insufficient for our goals on at least three accounts: First, existing work focuses only on selected performance parameters or overall device efficiency, rather than on providing a comprehensive coverage of the whole device sub-efficiencies for a particular LED product or design. Scientific publications also do not always disclose the underlying device architecture or the features responsible for the gains in performance. Second, not all relevant innovations are patented \cite{Pakes_1980,Fontana_2013}. In the case of LED patents in particular, our interviews with industry experts suggest that the propensity to patent is the highest for knowledge related to macroscopic device architecture and chemical composition of phosphors, and the lowest for knowledge related to manufacturing process improvements and microscopic chip architecture that is difficult to reconstruct by reverse engineering. This means that relying only on patent literature would bias results by unduly emphasizing some focus areas and de-emphasizing others. Third, scientific publications and patents typically focus on experimental devices, rather than commercial products. While new LED features, designs and manufacturing methods reported in these sources can potentially result in significant performance gains or cost reductions, it is difficult to ascertain if these improvements have since been adopted in industry.  Furthermore, information on LED manufacturing cost and the effect of process improvements on the total cost is highly proprietary. Estimates are occasionally reported in the scientific literature and company publications, but these often do not disclose which parts of the manufacturing process are responsible for the largest contribution to the overall cost, or which improvements led to cost reductions.

To overcome the limitations of these different methods for understanding technological progress, in this study we rely on a multi-method approach to data collection and analysis, the details of which are provided in Section 2 of the Supplementary Information document. Specifically, we combine information obtained from a systematic review of the primary scientific literature, device datasheets, relevant patents, and industry publications (SI Section 2.1) with information gained from semi-structured interviews with experts from academia and industry (SI Section 2.2), bottom-up manufacturing cost modelling (SI Section 2.3), and our own computations of device sub-efficiencies (SI Section 2.4). We then use this information to track the historical progress in white LED technology over time across the three groups of metrics identified above in the 'Metrics' section and identify its sources in innovation and technology spillovers.

\section{Methodology}
\label{sec:costmodel}

This subsection provides the details for the main methods and data sources used in our mixed-methods research approach, including the systematic literature review, semi-structured interviews with eminent experts, manufacturing cost modelling, and performance metrics calculations.

\subsection{Systematic Literature Review}

We collected data on LED performance and characteristics in a systematic literature review that included scientific publications, patents, conference proceedings from the largest semiconductor and optoelectronics conferences, industry periodicals and roadmaps, as well as company presentations and reports. This review was structured around the three main goals: 1) tracking the evolution of LED technology over time as indicated by three groups of progress metrics introduced in the main article; 2) identifying individual innovations that contributed to this evolution and whether or not they could be spillovers, and quantifying their impact on device performance and manufacturing cost; and 3) determining whether these innovations had originated within the LED technology domain, or in a field of research or technology outside of solid-state lighting, making them a technology spillovers. 

Relevant sources for the review were found in an iterative search process that involved two components. The first was the search in specialized patent and publication databases as well as company websites. The second component was the analysis of backward citations in the identified sources, starting from the reviews mentioned in section Previous Literature in the main article and then iteratively repeating it for all newly identified sources, until no further relevant and significant new sources were found. We also relied on backward citations for the identification of technology spillovers, considering cited documents as indicators of knowledge origins of an innovation and analyzing whether those documents belonged to the LED technology domain or not.

\subsection{Semi-Structured Interviews}

To supplement our data collection efforts, verify our findings and identify additional spillovers, we conducted a series of elite semi-structured interviews with eleven eminent experts from academia, industry and the public research sector. Experts were selected based on their engagement in  different sub-fields of LED research and manufacturing, as well as the recommendations from other interviewed experts, in essence expanding the list of experts that emerged from the initial literature review. All interviews were conducted between November 2019 and April 2022 by means of video conferencing and lasted for about one hour. A summary of the background of interviewed experts is provided in \cref{tab:interviews}.

The primary, structured part of the interviews explored which innovations were deemed most relevant to the evolution of device performance, consumer experience and manufacturing cost of LED packages. Thereafter, interviewees were asked to consider the extent to which these innovations may have originated outside of their respective field of expertise and the LED industry more broadly—i.e., which of the innovations may be considered spillovers. The remainder of the interview was focused on learning about particular aspects of the manufacturing processes relevant to cost and performance modelling, the current state of industry, and the circumstances surrounding the innovations and spillovers identified in the first part of the interview. Specific quantitative data was also provided by experts, helping fine-tune the parameters of the manufacturing cost model (described in \cref{sec:costmodel}) and verify device performance data.



\subsection{Performance Metrics Calculations}
\label{subsec:metrics}

The contribution of individual technology innovations and spillovers to the progress in overall device efficiency over time is estimated by index decomposition analysis. Mathematically, this involves breaking down a chosen performance indicator into its constituent components, each representing a specific factor that contributes to the change in the indicator \cite{Ang1997}. Specifically, we use the additive logarithmic mean Divisia index method I (LMDI-I), also known as the Additive Sato-Vartia indicator \cite{deBoer2019}. It was developed by Boyd in 1987 \cite{Boyd1987} on the basis of Divisia Index, a method in statistical economics \cite{Diewert1988}, and subsequently refined.

According to this method, for an overall device efficiency function $F$ that is the product of variables $a, b$ that represent sub-efficiencies, the contribution of the change in a single sub-efficiency variable $a$ between times $t=0$ and $t=T$ can be estimated as \cite{Ang2019}

\begin{align}
    \Delta a &= \frac{a_{t=T} - a_{t=0}}{\ln(a_{t=T}) - \ln(a_{t=0})} \times \ln \big ( \frac{a_{t=T}}{a_{t=0}} \big ) \\
    & \stackrel{a_{t=0} \neq a_{t=T}}{=} L(F_{t=T}, F_{t=0}) \times \ln \big ( \frac{a_{t=T}}{a_{t=0}} \big )
\end{align}

where $L(F_{t=T}, F_{t=0})$ is the logarithmic mean of $F$ values at times $t=0$ and $t=T$. These terms contain no residuals, therefore it can be shown that the overall improvement in the device efficiency due to improvements in individual sub-efficiencies is equal to the sum of these improvements in individual sub-efficiencies: 

\begin{equation}
    \Delta a + \Delta b  = \Delta F
\end{equation}

To document historical improvements in LED device performance accurately, we need data on all sub-efficiencies for the selected device architectures and periods covered. However, the scope of data reporting in scientific literature and industry publications is typically limited to selected metrics of interest, rather than the full ensemble of sub-efficiencies that determine the overall device performance. For this reason, our data collection efforts were supplemented by performance calculations for individual sub-efficiencies where possible and necessary. Of the device sub-efficiencies, those related to the emission spectrum  were computed from the spectral data often reported in LED device specifications. In particular, we used the \texttt{colour-science} package for Python\cite{python-colour} to calculate the luminous efficacy of radiation, colour rendering performance and luminous efficacy of radiation of phosphor down-conversion of blue light on the basis of available LED emission spectra. This approach allowed us to  quantify the improvements related to phosphor development in LEDs. 

\subsection{Manufacturing Cost Model}

\subsubsection{Our Approach to Cost Modeling}

The model structure is generally based on the 2012 LEDCOM cost model, but we expand it significantly both in scope and in its ability to capture historical trends. The model captures three historical time periods corresponding to different “eras” in LED manufacturing: the early period of the first high-power white LEDs around 2003, the period of accelerating consumer adoption of LED lighting around 2012, and the most recent period around 2020, the year of our main data collection efforts. For each of these three years, the most prevalent manufacturing equipment was identified through industry periodicals, archived website data from the \textit{Internet Archive}, and expert interviews. Because the architecture of LED chips has changed significantly since the introduction of the first commercial devices in 1996, three different chip architectures were initially considered in the model: classical chips, flip chips, and chip-scale package flip chips. The details of the manufacturing process for each architecture were collected from the scientific literature, textbooks and relevant patents. In addition, two LED life cycle analyses \cite{scholand2012life}\cite{casamayor2018comparative} were used to validate the model structure and extract some of the necessary quantitative model inputs. These studies captured a large number of LED manufacturing process steps and included the details on the use of metals, chemicals and electricity for each manufacturing step.

The aggregate result of the cost model is the manufacturing cost per LED package for each of the three years considered, which includes all costs associated with producing the chip, including running costs of the factory. Costs associated with research and development, administrative overhead of the manufacturer or other investment costs are not considered. We also note that the purpose of our cost model is not to give specific estimates of LED manufacturing cost for a factory of any size, specific geographic location or total annual manufacturing volume. It instead assumes an hypothetical factory with an assumed location in the United States and associated overhead costs related to the operation of the factory. It also assumes the use of the most up-to-date equipment for that year. Even with these simplifying assumptions, the model reasonably identifies the impact that changes in single process steps can have on the total LED manufacturing cost. An important limitation of our cost modelling efforts is that, even though the model captures three different chip architectures in its structure, in the present study we were able to collect, estimate and present the full set of quantitative inputs and outputs only for the classical chip architecture of low- to mid-power devices. Populating the model with data for the remaining two architectures requires access to proprietary industry information, which we have not been able to get thus far.  

The cost model we developed relies on a cumulative approach to yielded cost\cite{becker2001use}. In this approach, the yielded cost of process step $1$ is defined as the ratio between the total cost of step 1 $C_1$ and the yield of step 1 $Y_1$:

\begin{equation}
    C_{Y_1} = \frac{C_1}{Y_1}, \ C_{Y_2} = C_{Y_{2 \rightarrow 3}} - C_{Y_1} = \frac{C_1(1-Y_2)+C_2}{Y_1Y_2}, \ C_{Y_3}=\dots
\end{equation}

This cost metric is cumulative by definition, thus

\begin{equation}
    \sum_i C_{Y_i} = \frac{\sum_i C_i}{\prod_i Y_i}
\end{equation}

Yielded cost per step is dependent on the step order and blind to downstream information \cite{becker2001use}.

\end{document}