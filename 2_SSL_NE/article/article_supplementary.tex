\documentclass[parskip=full]{article}

% functionality
\input{auxiliary/preamble/1_preamble_functionality_main}

% layout
\input{2_SSL_NE/article/auxiliary/preamble/2_preamble_layout_paperdraft}

% nature energy custom code

\newcounter{supplementarynote}
\renewcommand{\thesupplementarynote}{\arabic{supplementarynote}}
\newcommand{\supplementarynote}{%
  \refstepcounter{supplementarynote}%
  Supplementary Note~\thesupplementarynote:~%
}
\titleformat{\section}
  {\normalfont\large\bfseries}
  {\supplementarynote}
  {0em} % Adjust this value to control the spacing
  {}
\renewcommand{\thesection}{\thesupplementarynote}

% bibliography
\usepackage[
    backend=biber,
    style=ieee
]{biblatex}

% show DOI URL (https://doi.org/XXX.XXXXX.XXXX), instead of publisher URL (https://springer.com/XXXX)
% compare https://tex.stackexchange.com/a/616241
\DeclareSourcemap{
  \maps[datatype = bibtex]{
    \map{
      \step[notfield = keywords, final]
      \step[fieldsource = doi, final]
      \step[fieldset = url, null]
    }
    \map{
      \step[fieldsource = keywords, notmatch = \regexp{\bprimary\b}, final]
      \step[fieldsource = doi, final]
      \step[fieldset = url, null]
    }
  }
}
\AtEveryBibitem{
    \clearfield{urlyear}
    \clearfield{urlmonth}
}

% add exact location of equation in source citation (eg. 'cf. [1] (1.123)')
% compare https://tex.stackexchange.com/a/668944
\usepackage{mathtools}
\newtagform{tagcite}{(}{)}
\newcommand*{\tagcite}[2]{%
    \renewtagform{tagcite}{(}{, cf.\@ \cite{#1} (#2))}
    \usetagform{tagcite}
}
\AfterEndEnvironment{equation}{\usetagform{default}}
\addbibresource{bibliography.bib}

%% CUSTOM SETUP %%%%%%%%%%%%%%%%%%%%%%%%%%%%%%%%%%%%%%%%%%%%

% custom figure and table numbering
\renewcommand{\thefigure}{Supplementary Figure \arabic{figure}}
\renewcommand{\thetable}{SI\arabic{table}}

% custom equation numbering
% https://tex.stackexchange.com/a/184641/
\newcounter{defcounter}
\setcounter{defcounter}{0}
\newenvironment{myequation}{%
\addtocounter{equation}{-1}
\refstepcounter{defcounter}
\renewcommand\theequation{SI\thedefcounter}
\begin{equation}}
{\end{equation}}

%% METADATA %%%%%%%%%%%%%%%%%%%%%%%%%%%%%%%%%%%%%%%%%%%%%%%%

\hypersetup{
    pdftitle={Supplement},
    pdfauthor={Michael Weinold, Sergey Kolesnikov, Laura Diaz Anadon},
}

%% MAIN DOCUMENT %%%%%%%%%%%%%%%%%%%%%%%%%%%%%%%%%%%%%%%%%%%

\begin{document}

\setlength{\fboxsep}{10pt}
\fbox{
    \parbox{\textwidth}{
        \textit{\textsc{Supplementary Information}} for: \\
        \newline 
        M. Weinold$^{1,2,3}$, S. Kolesnikov$^3$, L.D. Anadon$^{3,4}$ \\
        "Rapid technological progress in white light-emitting diodes and its sources in innovation and technology spillovers" (2023) \\
        \newline
        $^1$ Technology Assessment Group, Paul Scherrer Institute, Switzerland \\
        $^2$ Group for Energy Systems Analysis, Department of Mechanical and Process Engineering, ETH Zurich, Zurich, Switzerland \\
        $^3$ CEENRG, Dept. of Land Economy, University of Cambridge, UK \\
        $^4$ Belfer Center for Science and International Affairs, Harvard University, Cambridge MA, USA
    }
}

\tableofcontents

\newpage

\section{Previous Literature on Technological Progress in LEDs}
\label{sec:prev_lit}

The historical development development of light-emitting diodes from a novelty semiconductor experiment into powerful lighting devices has received a considerable amount of attention following the recognition of pioneering work on blue LED by Japanese researchers Shuji Nakamura, Isamu Akasaki and Hiroshi Amano with the 2014 Nobel Prize in Physics \cite{Nakamura2015}. The sources and disaggregated contributions of subsequent improvements in LED technology, however, have not been documented in the literature in a systematic fashion, though there were notable publications reporting on the progress in the design of devices \cite{Shchekin2006,krames2007led,laubsch2009high,hahn2014development}, or the technological improvements  underlying the improvement in overall device efficiency for best performers in 2009 \cite{tsao2010solid} and 2016 \cite{pattison2017solid}. 

Several descriptive publications provide a very useful overview of selected scientific breakthroughs that have contributed to LED progress \cite{krames2007status,Phillips2007,Bierhuizen2007,Nakamura2013,feezell2018invention,Taki2019}. Additional studies provide more detail regarding the origin and impact of specific advances in LED technology as well as their integration into device manufacturing were published by authors working for key industry actors Lumileds\cite{MuellerMach2005,Shchekin2006,lumi2015lumi,Bhardwaj2017} and Osram (now amsOsram)\cite{Haerle2004,Baur2009,laubsch2009high,hahn2014development}. However, these studies typically include limited information on the macro-level chip design and cover disparate aspects of the technology over different time periods. Another important source of historical information are patents, which cover the entirety of the device architecture or manufacturing process, including those by Lumileds\cite{margalith2011thin}, Samsung\cite{jung2014phos,cha2019semiconductor} and Soraa\cite{cich2017high}.

Despite the amount of literature published on the topic of LED history, we are not aware of a prior publication that comprehensively and consistently aggregates and analyzes known chip design, manufacturing, and material improvements to show the overall effect of these improvements on device efficiency or manufacturing cost over time. Existing publications, including those mentioned above, are typically subject to one of the following key limitations:

\begin{itemize}
    \item Publications often look only at progress across a select few LED device sub-efficiencies, often during a very short time period and, most importantly, only at devices manufactured by one specific company, e.g., OSRAM Opto Semiconductors in \cite{hahn2014development} or Lumileds in \cite{Bhardwaj2017}.
    \item Publications do not report on a number of potentially sensitive or proprietary advances, especially following the market entry of Chinese LED manufacturing companies as a result of the 12th Five-Year Plan (cf. \cite{guo2017china}).
    \item Publications typically do not look at the overall LED efficiency and its underlying physical limits and tend to overlook the metrics of progress related to consumer experience. 
\end{itemize}

Due to these limitations, any recommendations on future improvements in LED technology based on these studies are fragmented at best. Furthermore, the effect of individual innovations and technology spillovers, which has been investigated in solar PV \cite{kavlak2018evaluating,kolesnikov2020novel,nemet2019solar}, and to some extent in lithium-ion batteries \cite{Stephan_2021}, has not yet been studied in the context of lighting.

In our study, we overcome the limitations of previous literature by, first, synthesizing data provided in multiple publications across an ensemble of various cost, performance and consumer experience metrics for tracking progress in LED technology over an extended period of time. Second, our key advantage and contribution compared to these publications is a multi-method approach that we take to fill in the data gaps and augment information obtained from the literature sources by our own performance calculations, a bottom-up cost model and a series of expert interviews. This enabled us to synthesize evidence from multiple sources and, for the first time, reconstruct a comprehensive picture of the historical progress in white LED technology throughout its entire history.

\section{Metrics for Technological Progress in LEDs}

This Supplementary Note provides additional background and details for the physical, economic and consumer experience metrics that were chosen to quantify multi-dimensional progress in light-emitting diode technology.

\subsection{Overview}

Investigating the sources of rapid progress in LED technology over the past decades, in which white LEDs have come to dominate the lighting market \cite{zissis2021}, requires selecting appropriate metrics for tracking and quantifying this progress. The choice of metrics affects both what data sources can be used in the analysis and which research methodologies can be used to calculate and analyse such metrics. We select the metrics based on the following two general criteria. First, we focus only on metrics widely accepted and reported in industry because metrics proposed in the scientific literature but not reported by device manufacturers cannot be used to compare the performance of commercial LED devices over time. Second, the chosen metrics must be useful for understanding the impact of individual technological improvements on relevant performance and cost characteristics.

Historically, the progress in LED technology has been commonly described by pointing to impressive improvements in LED device performance, more specifically device brightness, as well as electrical efficiency, and manufacturing cost reductions \cite{Taki2019}. However, metrics related to consumer experience from using white LEDs as a lighting source, such as the perceived temperature of a white light source and its ability to faithfully render colours, have also played a significant part in the market adoption of new lighting technologies \cite{Menanteau2000,Sandahl2006,CAIRD2008,murphy2012governing} and received substantial attention in LED research and development efforts \cite{azevedo2009transition,cho2017white}. Therefore, a comprehensive analysis of the evolution in white LED technology must take into consideration advances in 1) physical device performance; 2) consumer experience; and 3) LED device manufacturing costs. Next, we introduce and discuss the metrics that we use to track progress in each of these three areas.

\subsection{Device Performance Metrics}
\label{sec:device_performance_metrics}

During the years following the market introduction of white LEDs, the primary metric of progress in solid-state lighting was typically luminous flux. This was because the luminous flux (total brightness) of early white LEDs was too small to allow for the economical combination of multiple LEDs into lamps for general illumination purposes. In 2000, the highest performing devices yielded around 10lm, just below the output of a candle as defined in the unit candela (1cd=12.57lm)\cite{haitz2011solid}. Progress in this metric was commonly rendered together with the associated reduction in retail prices as “Haitz’s Law” \cite{haitz1999case,haitz2011solid}. Today, however, the devices with the highest performance yield in excess of 1600 lm, the equivalent of a 100 W incandescent bulb \cite{cree2020bright}. LED brightness has thereby become sufficient to enable the construction of lamps from multiple LED devices, with contemporary improvements focusing instead on higher efficiency and quality of light instead of brightness. Even though a large number of scientific publications and industry periodicals continue to focus on brightness as a metric of progress in lighting, we find that, at this point in time, this metric is insufficient to capture the complexity of the multitude of efficiency improvements that have been driving the overall LED efficiency \cite{weinold2021compound}. Specifically, the stagnating levels of luminous flux in the highest performing devices do not capture the outcomes of a major area of LED research, which is improving electrical efficiency at constant brightness.

We therefore use a combination of the total device efficiency ("lamp efficiency") and the sub-efficiencies that describe the different physical loss channels within the device to describe progress in light-emitting diode technology. In line with previous publications \cite{schubert2018light,tsao2010solid}, we use the sub-efficiencies: forward voltage efficiency $\eta_{V_f}$, light extraction efficiency $\eta_{LE}$, internal quantum efficiency $\eta_{IQ}$, droop $\eta_{droop}$, conversion efficiency $\eta_{C}$, spectral efficiency $\eta_{S}$. The equations defining these sub-efficiencies are provided in the Supplementary Information in Section 1.3. The overall efficiency ("Lamp Efficiency") $\eta_L$ of a light-emitting diode package is the product of all considered sub-efficiencies:
%
\begin{equation}
    \eta_L = \prod_{i=(V_f, \dots, S)} \eta_i
\end{equation}
%
This metric describes the cumulative electrical and optical losses within the device, as well as the light conversion losses in the phosphor layer.

\subsubsection{Forward Voltage Efficiency $\eta_{V_f}$}

Device forward voltage efficiency\footnote{"Joule Efficiency" $\epsilon_J$ in Tsao et al. \cite{tsao2010solid}} (FVE) describes all electrical losses at the interface between the electrodes and the semiconductor and in the bulk. These losses can be due to tunneling and Ohmic resistance at the interface, as well as Ohmic resistance and other electrical losses within the bulk of the semiconductor. It is defined as
%
\begin{equation}
    \eta_{V_f} = \frac{E_{h\nu}}{V_f}
\end{equation}
%
where $E_{h\nu}$ denotes the photon energy and $V_f$ is the diode forward voltage \cite{schubert2018light}\cite{tsao2010solid}.

\subsubsection{Light Extraction Efficiency $\eta_{LE}$}

LED light-extraction efficiency (LEE) describes the  losses due to absorption in the material after electron-hole recombination and the associated emission of a photon. It is defined as
%
\begin{equation}
    \eta_{LE}=\frac{\text{\# of photons out}}{\text{\# of photons created}} = \frac{P_{opt}}{P_{int}}
\end{equation}
%
where $P_{opt}$ is the optical power of the device and $P_{int}$ is the internal power of the device\cite{schubert2018light}\cite{tsao2010solid}.

\subsubsection{Internal Quantum Efficiency $\eta_{IQ}$}

Internal quantum efficiency (IQE) describes non-radiative recombinatory processes in the semiconductor bulk. It is related to the external quantum efficiency through
%
\begin{equation}
\label{eqn:ieq-eqe}
    \eta_{EQ} = \eta_{IQ} \times \eta_{LE}
\end{equation}
%
It is defined as
%
\begin{equation}
    \eta_{IQ} = \frac{\# \text{of photons created}}{\# \text{of electron-hole pairs in}}
\end{equation}
%
and depends on the current density of the device\cite{schubert2018light}\cite{tsao2010solid}. 

\subsubsection{Droop $\eta_{droop}$}

Efficiency droop describes the decrease of LED internal quantum efficiency at high current densities, which is caused by a number of different physical effects\cite{David2020}. This energy loss channel is often treated separately from internal quantum efficiency in the literature due to its importance in high-power devices. It is defined as
%
\begin{equation}
    \eta_{droop} = 1 - \frac{\eta_{IQ}}{\eta_{IQ}(A \rightarrow 0)}
\end{equation}
%
where $\eta_{IQ}(A \rightarrow 0)$ denotes the internal quantum efficiency at low current densities. In practice, droop is often given as the percentage difference between the ideal luminous intensity curve $\phi_{ideal}$ and the real luminous intensity curve $\phi$ at a set diode test current $A_{test}$
%
\begin{equation}
\label{eqn:droop}
    D = \frac{\phi_{ideal}(A_{test})-\phi(A_{test})}{\phi_{ideal}(A_{test})/100}
\end{equation}
%
Therefore, according to this definition, a lack of droop corresponds to a droop efficiency of $\eta_{droop} = 100\%$\cite{schubert2018light}\cite{tsao2010solid}.

\subsubsection{Light Conversion Efficiency $\eta_{C}$}

Light conversion efficiency (LCE) describes the losses in the conversion process of blue light that is the basis for all phosphor-converted white LEDs. These losses are the sum of the Stokes loss and well as scattering/absorption losses. It is defined as
%
\begin{equation}
    \eta_{C} = \frac{E_{\textcolor{blue}{B}}}{\sum_{i=\textcolor{red}{R},\textcolor{orange}{O},\textcolor{yellow}{Y},\textcolor{teal}{G}} E_i}
\end{equation}
%
where $E_i$ denotes the total energy of light at the color $i$ corresponding to the down-converted photon wavelength \cite{schubert2018light}\cite{tsao2010solid}. Since every individual phosphor component in the device has its own associated conversion losses, the denominator sums over all phosphor components, i.e. \textcolor{red}{R}ed, \textcolor{orange}{O}range, \textcolor{yellow}{Y}ellow, \textcolor{teal}{G}reen.

\subsubsection{Spectral Efficiency $\eta_{S}$}

Spectral efficiency (SE) describes losses in the conversion process due to the wavelength-dependent efficiency of the human eyes. Photons converted into the infrared or ultraviolet are lost to illumination purposes. It is defined as

\begin{equation}
    \eta_{S} = \frac{K}{K_{max}(CRI,CCT)}
\end{equation}

where $K$ is the luminous efficacy of radiation of the light source, which can be computed from the device spectrum and the luminosity function, which describes the sensitivity of the human eye. $K_{max}$ is the maximum luminous efficacy of radiation of a perfect light source with the same color rendering performance and correlated color temperature as the light source in question\cite{schubert2018light}\cite{tsao2010solid}.


\subsection{Luminous Efficacy of Radiation}
\label{subsec:ler}

Figure 1 in the main publication uses luminous efficacy of radiation as the primary metric to describe progress in lighting technologies. Care must be taken not to confuse this metric with luminous efficacy of source, which is used in \cref{fig:phosphor_spectrum}.

This metric describes the match of a light-emitting diode package spectrum to the human visual system. Efficacy in lighting is dependent on the luminosity function, which describes the wavelength-dependent sensitivity of the human eye. A light source emitting very \textit{efficiently} in the infrared yet emitting no visible light has a very low \textit{efficacy}. The luminous efficacy of radiation $K$ is mathematically defined as the normalized, integrated product of the spectral radiant flux of a light source with the wavelength-dependent human sensitivity to light \cite{cie-term-effrad}

\begin{myequation}
\label{eqn:ler}
    K [\text{lm/W}_{opt}]= \frac{\int_0^\infty K( \lambda ) \phi \text{d} \lambda}{\int_0^\infty \phi \text{d} \lambda}
\end{myequation}

where

\begin{align*}
    K &\dots \text{spectral luminous efficacy} \\
    \phi &\dots \text{spectral radiant flux} \\
    \lambda &\dots \text{wavelength}
\end{align*}

This metric can be computed from spectral data alone and does not require additional spectral normalization. It enables straightforward comparison between the performance of different downconversion phosphors, as shown in the top panel of \cref{fig:phosphor_spectrum}. Light sources emitting in the far red or blue part of the spectrum have lower efficacy of radiation. Care must be taken not to confuse this efficacy metric with the \textit{efficacy of source} described in the following subsection.

\subsection{Luminous Efficacy of Source}
\label{subsec:les}

The luminous efficacy \textit{of a light source} $\eta$ is defined as the ratio between the emitted luminous flux and the consumed electrical power \cite{cie-term-effsrc}

\begin{myequation}
    \eta [\text{lm/W}_{el}]= \frac{\phi}{P_{el}}
\end{myequation}

This metric is often cited in device datasheets, scientific literature and textbooks when describing the performance of light-emitting diodes. Care must be taken not to confuse this efficacy metric with the \textit{luminous efficacy of radiation}, which depends only on the spectral characteristics of a light source. As the luminous efficacy of a light source $\eta$ captures the overall device efficacy, it depends on a large number of other device properties and parameters. This makes attribution of changes in this metric to individual changes in device design or manufacturing difficult. For this reason, we do not use this metric in our study.

\subsection{Consumer Experience Metrics}

The perceived quality of light is entirely determined by the emission spectrum of a light source \cite{ies_handbook}. Any metric relevant to customer experience can thus be calculated from the spectrum alone. The spectrum of an LED light source is determined by the emission wavelength of the LED itself and the absorption and emission spectra of the down-conversion phosphor used in the device. It is typically included in the product datasheets provided by manufacturers, which enables the calculation of all relevant spectrum-based metrics for these devices. Based on the prevalence in scholarly literature and industry publications, for this study we choose two consumer experience metrics: colour rendering index (CRI) and colour temperature (CT). We do not consider flicker, the unintended high-frequency temporal modulation of light, which is another important consumer experience metric for lighting and a subject of recent regulation by the European Union \cite{weinold2020long}. This effect is caused not by LEDs themselves, but rather by inadequately designed electrical ballasts \cite{Lehman2014}. As a result, it is beyond the scope of this work. 

\subsubsection{Color Rendering Index (CRI)}

The colour rendering index (CRI) of a light source describes its ability to render the colours of an object faithfully when compared to illumination under a reference light source, such as standard daylight \cite{khan2015led}. The way it is calculated is defined by the International Illumination Commission (CIE) \cite{cie_cri_1995}. CRI has certain limitations when applied to solid-state light sources \cite{david2013cri}. \cite{Houser2013}, CRI has remained the de facto industry standard for describing colour fidelity of light sources \cite{DOE2016LED}. High colour rendering performance of lighting is a requirement in workplace environments, retail stores, clinical operating environments and art exhibitions \cite{khanh2017color}. It should be noted that some niche applications prioritize high colour saturation over high CRI, for instance, in food display or fabric retail applications \cite{david2013cri}. However, these niche applications remain outside our focus on general illumination. Due to a broad availability and importance of CRI data for consumers of various LED lighting sources, we adopt CRI as the key metric to track progress in consumer experience in LED lighting, despite its limitations.

\subsubsection{Colour Temperature}

The colour temperature (CT) of a light source describes the equivalent temperature of an ideal black body which emits light of a colour comparable to that of the light source \cite{commission2011cie}. Warm white light sources are widely used in general illumination, while cold white light sources are used in workplace illumination and outdoor lighting. Early white LEDs produced only cool white light \cite{mueller2000light}. The introduction of first commercial warm white LED light bulbs played a significant role in increasing adoption of LED-based lighting among consumers, as their spectrum more closely resembled the warm white colour temperature of incandescent light bulbs \cite{al2016optics}. For this reason, we also adopt CT as a metric for tracking progress in consumer experience in LED lighting.

\subsection{Cost Metrics}

In selecting metrics for tracking the progress in manufacturing cost reductions in LED, we must highlight the complexity of this task. Access to manufacturing cost data at the chip level is usually restricted due to its proprietary nature and is available only for selected products. Using sales price information instead of the cost for the same purpose seems a promising alternative, as prices can be easily obtained directly from manufacturers for current products. However, historical data on prices for different chip architectures and different years is similarly difficult to obtain. In addition, sales price includes components not relevant to progress in the technology, such as profit margins and overhead costs, and is affected by policies such as rebates and purchase subsidies. Historical information on these factors and price components, as well as their impact on the manufacturing cost for each LED product under consideration is even harder to obtain, making the use of LED sales prices as a direct metric of technology progress very difficult in practice. 

We address the limitations of data availability for the chip-level LED manufacturing costs and sales prices by developing and applying a bottom-up LED manufacturing cost model with process-step resolution. In our case, it is enabled by the general availability of historical data on prices of relevant raw materials, components, and manufacturing equipment, and the relative cost data on manufacturing processes, which is occasionally published as part of industry press releases \cite{ledinside2013csp}\cite{seoul2015csp}. 

We describe the details of our manufacturing cost model, including its structure and equations, manufacturing process flows for the chip architectures under consideration, input data, explanation of our cost modelling approach based on yielded costs, as well as the model's limitations, in Supplementary Information section 2.4. The model structure is generally based on the 2012 LEDCOM cost model \cite{ledcomv2}, but we expand it significantly both in scope and in its ability to capture historical trends. The model captures three historical time periods corresponding to different “eras” in LED manufacturing: the early period of the first high-power white LEDs around 2003, the period of accelerating consumer adoption of LED lighting around 2012, and the most recent period around 2020, the year of our main data collection efforts. The aggregate result of the cost model is the manufacturing cost per LED package for each of the three years considered, which includes all costs associated with producing the chip, including running costs of the factory.


\clearpage
\section{Methodological Details}

This Supplementary Note provides additional details for the main methods and data sources used in our mixed-methods research approach described in the Methods section in the main text.

\subsection{Background of Interviewed Experts}

\begin{table}[H]
\small
    \centering
    \caption{Anonymized list of interviewed LED experts.}
    \vspace{5mm}
    \begin{tabular}{|l|l|l|l|l|}
    \hline
        \textit{\#} & \textit{Sector} & \textit{Role} & \textit{Country} & \textit{Expertise} \\ \hline
        1 & Academia & Senior researcher & UK & Epitaxy \\ \hline
        2 & Industry & Consultant, former senior researcher & USA & Device architecture \\ \hline
        3 & Industry & Consultant, former head of R\&D & Germany & Epitaxy \\ \hline
        4 & Academia & Professor & Austria & Phosphors \\ \hline
        5 & Industry & Consultant, former head of R\&D & USA & Device architecture \\ \hline
        6 & Consulting & Consultant, former senior technical advisor & USA & Device architecture \\ \hline
        7 & Academia & Professor & Germany & Phosphors \\ \hline
        8 & Government & R\&D manager & USA & Device architecture \\ \hline
        9 & Consulting & Consultant & USA & Device applications \\ \hline
        10 & Academia & Professor & France & Device physics \\ \hline
        11 & Industry & Senior scientist, former head of R\&D & USA & Device architecture \\ \hline
        12 & Industry & Principal scientist & Germany & Phosphors \\ \hline
        13 & Industry & Former head of R\&D & USA & Phosphors \\ \hline
    \end{tabular}
    \label{tab:interviews}
\end{table}

\clearpage
\section{Additional Results}

This section provides additional background and details for the results of our study presented in the Results section in the main text.

\subsection{Historical Progress in Sub-Efficiencies}

The historical progress in the different sub-efficiencies contributing to overall efficiency (lamp efficiency) is shown in \cref{fgr:subeff}. Context for figure panels is provided below:

\textbf{Panel A1:} Device forward voltage at a test current of I=350mA. The physical limit for a blue light wavelength of 450nm without electric pumping is shown for reference. Data points for devices released in 2020 by various manufacturers are shown in an inset plot. \

\textbf{Panel A2:} Efficiency droop at the test current of I=350mA. \

\textbf{Panel B1:} Internal quantum efficiency, for different chip architectures, by type of measurement used. Note that the artifact in the DOE Average around 2013 is due to a change in definition for internal quantum efficiency laid out in \cite{doe_ssl_multiyear_2013}. Measurement methods: PL - Photoluminescence\cite{Shim_2018}, EL - Electroluminescence\cite{Getty_2009};  ABC Model\cite{Karpov_2014}, Ideality Factor\cite{Masui_2010}. “Not reported” denotes data points where the IQE measurement technique was not reported. Whiskers indicate reported ranges, where applicable. \

\textbf{Panel B2:} Light extraction efficiency for different chip architectures, by type of specific technology used to improve light-extraction efficiency. The source of data points is shown for reference: simulation - ray-tracing computer simulation; experimental (absolute) - light-extraction efficiency given directly; experimenal (relative) - relative improvement over baseline chip architecture given. Abbreviations: TFFC - Thin-Film Flip-Chip; FC - Flip-Chip; CSP - Chip-Scale Package; ITO - Indium Tin Oxide; ZnO - Zinc Oxide; PSS - Patterned Sapphire Substrate; electr. - electrode; text. - textured. 

\begin{figure}[h!]
 \centering
 \includegraphics[width=15.5cm]{./figures/subefficiencies_progress.pdf}
 \caption{Historical improvements in light-emitting diode technology. Panels A1-A2 show they key metrics used to compute forward voltage efficiency $\eta_{Vf}$ and droop efficiency $\eta_{droop}$. Panels B1-B2 show the sub-efficiencies internal quantum efficiency $\eta_{IQ}$ and light extraction efficiency $\eta_{LE}$. Source: own synthesis of data from the full list of sources provided in \cref{sec:sources}. Data on average device performance adapted from U.S. Department of Energy (DOE) Reports \cite{doe_ssl_multiyear_2007}\cite{doe_ssl_multiyear_2008}\cite{doe_ssl_multiyear_2013}\cite{doe_ssl_rnd_2016}\cite{doe_ssl_rnd_2018} and an Osram company report \cite{beale_leds_2015}.}
 \label{fgr:subeff}
\end{figure}

\clearpage
\subsection{LED Innovations}

\subsubsection{Non-Phosphor-Related}

\cref{tab:innovations} summarizes LED innovations and technology and manufacturing process improvements identified in this study as affecting key white LED sub-efficiencies. Note that phosphor-related LED innovations affecting consumer experience metrics, along with spectral efficiency, are presented separately in Table 1 in the main publication and discussed in detail in \cref{sec:innovation_phosphor}.

\begin{table}[h!]
    \caption{LED innovations and technology improvements affecting key white LED device sub-efficiencies. The table includes innovations and technology improvements affecting forward voltage efficiency, light extraction efficiency, internal quantum efficiency, and light conversion efficiency. For a list of innovations affecting spectral efficiency, see Table 1 in the main article.}
    \vspace{5mm}
    \begin{NiceTabularX}{\textwidth}{ |l|X|X|l|l| }
        \multicolumn{5}{c}{Forward Voltage Efficiency (VFE) $\eta_{Vf}$} \\
        \hline
            \textit{Year} & \textit{Innovation} & \textit{Area of Improvement} & \textit{Spillover} & \textit{Source} \\
        \hline
            1999 & Indium Tin Oxide \newline Current spreading layer & Contact resistance & Yes & \cite{margalith1999indium}\\
        \hline
            1998 & Digitated electrodes & Contact resistance & No & \cite{steigerwald2001electrode} \\
        \hline
            Ongoing & Epitaxy improvements \newline and better doping & Polarization and \newline bulk resistance & No & I \\
        \hline
            1998 & Silver p-Contacts & Contact resistance & No & \cite{kondoh2001nitride} \\
        \hline
        \multicolumn{5}{c}{Light extraction efficiency (LEE) $\eta_{LE}$} \\
        \hline
            \textit{Year} & \textit{Innovation} & \textit{Area of improvement} & \textit{Spillover} & \textit{Source} \\
        \hline
            $<$ 2003 & Optimization for \newline cavity effects & Reduces self-interference of quantum well & No & I, \cite{Shen2003} \\
        \hline
            1993 & Chip surface randomization & Total reflection and absorption & No & I, \cite{bergh1973surface}\cite{Schnitzer1993} \\
        \hline
            1993 & Thin-film chip architecture & Absorption & No & I, \cite{Schnitzer1993} \\
        \hline
            1996 & Patterned sapphire substrate  & Total reflection and absorption & Yes & I, \cite{Tadatomo2001} \\
        \hline
            Ongoing & Chip design for high LEE & Total reflection and absorption & No & I, \cite{Haerle2004} \\
        \hline
            $\sim$2000 & Silver p-contacts & Absorption & No & I, \cite{kondoh2001nitride} \\
        \hline
        \multicolumn{5}{c}{Internal quantum efficiency (IQE) $\eta_{IQ}$} \\
        \hline
            \textit{Year} & \textit{Innovation} & \textit{Area of improvement} & \textit{Spillover} & \textit{Source} \\
        \hline
            1994 & Double heterostructure & Higher Radiative recombination probability & No & \cite{Nakamura1994} \\
        \hline
            1996 & Multiple quantum well & Higher Radiative recombination probability & No & \cite{Koike1996} \\
        \hline
            Ongoing & Active region \newline Doping & Radiative recombination probability & No & I, \cite{schubert2018light} \\
        \hline
            Ongoing & Epitaxy \newline Improvements & Radiative recombination probability & No & I \\
        \hline
            Ongoing & Chip architecture \newline Improvements & Radiative recombination probability & No & I \\
        \hline
        \multicolumn{5}{c}{Light Conversion efficiency (CE) $\eta_{C}$} \\
        \hline
            \textit{Year} & \textit{Innovation} & \textit{Area of improvement} & \textit{Spillover} & \textit{Source} \\
        \hline
            Ongoing & Lower current density & Current density & No & I \\
        \hline
            Ongoing & Epitaxy \newline Improvements & Charge distribution \newline Defect density & No & I, \cite{bhardwaj2016progress} \\
        \hline
            Ongoing & Chip architecture \newline improvements  & Charge distribution \newline Defect density & No & I, \cite{Wildeson2017} \\
        \hline
            $<$ 2017 & Defect getting \newline underlayer & Defect Density & No & I, \cite{haller2017burying} \\
        \hline
        \end{NiceTabularX}
    \vspace{5mm}
    
     Note: \textit{Year} column indicates the first instance of application of corresponding invention in white LEDs. 'Ongoing' indicates improvements that are incremental in nature and have been ongoing since the earliest days of LED manufacturing, with no individual breakthroughs identified. \textit{Spillover} column indicates if the innovation involved technology spillovers. \textit{Source} column indicates the source of information about the innovation or improvement, with 'I' indicating expert interviews as such a source.
    \label{tab:innovations}
\end{table}

\subsubsection{Phosphor-Related}
\label{sec:innovation_phosphor}

This section provides additional details for phosphor-related LED innovations identified in this study as affecting both spectral sub-efficiencies and consumer experience metrics.

\paragraph{YAG and YGAG Phosphors}

Prior to Nakamura’s invention of highly efficient blue LEDs in the 1990s, Japan’s Nichia Corporation had not sold commercially successful semiconductor products, instead specializing in phosphors for cathode ray tubes (CRT) and fluorescent lamps \cite{nakamura2013blue}  . Nevertheless, extensive firm expertise in this area helped Nichia’s Yoshinori Shimizu formulate the principles of using CRT phosphors to convert blue light from Nakamura’s LEDs into white light in 1994 \cite{shimizu1994sheet}\cite{cho2017white}. By 1996, Shimizu and his colleagues developed \cite{bando1996}\cite{shimizu1999light} the first practical LED application of a well-known Yttrium Aluminium Garnet (YAG) CRT phosphor activated with cerium \cite{blasse1967new}, enabling the first commercial white LED products manufactured and sold by Nichia since late 1996 \cite{bando1998development}\cite{cho2017white}. 

Importantly, the YAG phosphor does not exhibit the spectral properties desirable in general illumination applications (see Figure 6 in the main article). An early solution to this problem, which was first discovered in the late 1967 \cite{holloway1969optical} and suggested for LEDs by the same team at Nichia in 1996 \cite{bando1998development}\cite{shimizu1999light}, was to use the gadolinium-doped red-shifted YAG phosphor (YGAG). Used in combination with red-emitting sulfide phosphors, by 2002 it helped bring to the market the first generation of warm white light LED products, e.g., those produced by Lumileds \cite{Mueller2002}. However, sulfide phosphors led to accelerated deterioration of sensitive parts of LED devices and became less efficient as operating temperatures increased. New chemically stable and non-toxic red phosphors were needed. 

\paragraph{258, SLA and SALON phosphors}

In 1997, Hubert Huppertz and Wolfgang Schnick, working at the University of Bayreuth in Germany, synthesized the first compound in a new class of rare earth nitridosilicate materials \cite{Huppertz1997} later dubbed “258” due to a proportion of elements in its chemical formula. The luminescent properties of these materials were identified by the Schnick’s group, by then at Ludwig-Maximilians University of Munich, in 2000 \cite{Hppe2000} after a suggestion made to Schnick at a conference following earlier reports of good luminosity properties of europium-doped compounds \cite{Qiua1998}. U.S.-based LED manufacturer Lumileds applied for a patent for the first class of red LED phosphors based on the 258 nitridosilicate chemistry in 2002 \cite{mueller2004phosphor}. The first use of the 258 phosphor in a commercial “Luxeon” LED package was then reported in a joint publication co-authored by inventors from Lumileds and researchers from the Schnick’s group in 2005 \cite{MuellerMach2005}.

Further efforts in LED phosphor development were directed towards synthesizing a red narrow-band phosphor. Narrow LED emission peak widths yield the highest luminous efficacy of radiation, as in this case less light is emitted in the far-red range of the spectrum in which the human eye is not very sensitive. After synthesizing several narrow-band phosphors emitting in yellow \cite{Hppe2004} and cyan \cite{Kechele2009}, the Schnick’s group identified the local cubic cation coordination structure of the cyan phosphor compound as the reason for its narrow band width \cite{lumi2016narrow}. A search for a structurally analogous nitride compound with the narrow red instead of the cyan emission was undertaken. After several unsuccessful attempts, the sought-after cuboidal nitride compound was found in a 2008 publication led by Francis DiSalvo \cite{Park2008Sr}. Based on information provided in this work, Schnick and colleagues synthesized and studied the spectral properties of a new narrow band red SLA phosphor in 2013 \cite{schmidt2013new}\cite{Pust2014}\cite{schmidt2017phosphors}. The material was introduced in commercial LED devices by Lumileds in 2015 \cite{lumi2016narrow_whitepaper}. 

The most recent red narrow-band phosphor innovation included in Table 1 and Figure 6 in the main article, indicated as SALON, has been under development during the late 2010s by a group of Austrian and German researchers that included Huppertz, the discoverer of the “258” material, working in collaboration with Osram, another major LED manufacturer \cite{seibald2019phosphor}\cite{Hoerder2019}\cite{Hoerder2020}. The first U.S. patent application for this phosphor was filed in 2017 \cite{seibald2019phosphor}. The SALON phosphor is a derivate of the SLA phosphor. Therefore, it is the only innovation related to consumer experience metrics identified in our study that seemingly not involved technology spillovers.

\paragraph{PFS Phosphor}

Down conversion with ultra-narrow-band phosphor can achieve the highest spectral efficiency. However, few such phosphors have been identified, with even less exhibiting desirable material properties such as thermal stability \cite{Phillips2007}. The first commercially successful ultra-narrow-band red phosphor was developed by General Electric (GE). It is based on a potassium fluorosilicate (PFS) compound activated with manganese ions. Its luminescence was first recorded by Adrian Paulusz at GE in 1972 \cite{paulusz1973efficient}. In the early 2000s, while searching for potential new LED phosphor materials for GE’s lighting business at GE Lumination, Emil Radkov rediscovered Paulusz’s findings in the literature. Following extensive research on PFS chemical synthesis and material properties conducted in collaboration with the University of Sofia in Bulgaria, Radkov’s Alma Mater, the PFS phosphor had been under development at GE since 2005 \cite{radkov2006red}\cite{radkov2009red}. This work, supported by public funding from the U.S. Department of Energy (DOE) Solid-State Lighting program \cite{doesslprogram}, resulted in a series of critical improvements in the PFS phosphor properties \cite{Setlur2010}\cite{lyons2012color}, eventually enabling its commercialization under the “TriGain” brand in 2015 \cite{trigain_spectrum}\cite{setlur2015trigain}\cite{Murphy2015}.

\paragraph{Quantum Dots for Light Down-Conversion}

Quantum dots (QD) are semiconductor nanocrystals whose quantum size effects make QDs behave as “artificial atoms”. Semiconductor quantum dots were first synthesized in the Soviet Union in 1981 \cite{ekimov1981quantum} and at Bell Labs in the U.S. in 1983 \cite{Rossetti1983}. Luminescent properties of quantum dots were first empirically observed in 1984 \cite{fojtik1984photo} and extensively studied in the early 1990s. The key feature of QD luminescence discovered in those studies is that its colour is determined by the QD particle size, making it possible to create pure monochromatic blue, green and red light sources just by tuning the QD size. The first application of QDs in LEDs was reported in 1994 in an electroluminescent hybrid QD-polymer LED. However, this LED type could not be used in general illumination due to its very low luminous efficacy. An alternative application of QDs as a kind of a “phosphor” for light down conversion from an LED light source was proposed in the early 2000s as part of the U.S. Department of Energy (DOE)-funded “A Revolution in Lighting“ project at Sandia National Laboratory \cite{simmonsfinal}. This concept was successfully demonstrated by Sandia researchers on a commercial LED in 2003 \cite{shea_rohwer_development_2004}\cite{noauthor_sandia_nodate} and was swiftly taken up and advanced further by a group in Taiwan \cite{Chen_2005}\cite{Hsueh_Shih_Chen_2006} The first commercial application of QDs in an LED lamp was brought about by a collaboration between an MIT-born startup QD Vision and the U.S.-based luminaire manufacturer Nexxus Lighting in 2009 \cite{ledprof_nexxusqd}, \cite{bourzac2013quantum}. However, rapid advances in the spectral and conversion performance of down-conversion phosphors and high manufacturing cost of quantum dots resulted in the discontinuation of this product. After finding market success in display backlighting first demonstrated by Samsung in 2010 \cite{Jang2010} and commercialized by QD Vision in Sony television sets in 2013 \cite{bourzac2013quantum}, QDs returned to the general lighting market in products offered by Lumileds \cite{noauthor_global_2017}\cite{noauthor_quantum_2020} around 2017 and Osram in 2019 \cite{osramqdots} in the form of mid-power LED packages that combined QDs with traditional phosphors for light down conversion.

\clearpage
\paragraph{Spectral Data for Identified LED Phosphor Innovations}

\hspace{1mm}

\begin{figure}[H]
	\centering
    \includegraphics[width=0.9\textwidth]{./figures/phosphor_spectrum-comparison.pdf}
	\caption{Spectral data and additional consumer experience metrics for the earliest identified representative white LED products with published spectral data, shown in Figure 6 in the main article, that used phosphor innovations listed in Table 1 in the main article, each indicated by the phosphor label and spectral data publication year. Top panel: Luminous efficacy of radiation (LER) and colour rendering index (CRI) of white LED devices represented in Figure 6 in the main article. The desirable direction of improvements towards higher luminous efficacy at higher CRI is indicated by a red arrow. Metrics were calculated from spectral data shown in the two bottom panels using the \texttt{colour-science} package for Python \cite{colour-science_software}. Bottom two rows of panels: Corresponding spectral data. The luminosity function \cite{cie-term-lumeff}, describing the wavelength-dependent sensitivity of the human eye, is shown for reference in red in each panel. Note that peaks or large tails of the device emission spectrum at the far ends of the luminosity function are not desirable, as the photons of the corresponding energy are lost to the human eye and count towards the spectral loss channel. Plot legends indicate the years of publication of the spectral data and phosphor mixtures used in corresponding LED devices, with the following designations for additional phosphor mix components: ? -  other parts of phosphor mixture not disclosed, G$^*$ - \ch{CaSrS:Eu^{2+}}, Y$^\star$ - $\beta-$SiAlON:Eu$^{2+}$, G$^\dagger$ - \ch{Lu_3Al_5O_{12}:Ce^{3+}}, O$^\ddagger$ - \ch{(Ba,Sr)_2Si_5N_8:Eu^{2+}}. Source (top panel): own elaboration based on spectral data. Arb. Units - Arbitrary Units, defined as the ratio of radiation intensity at a wavelength compared to the highest point in the spectrum. Sources: top panel - own elaboration based on spectral data; bottom two panels: adapted from published spectral data for LEDs with the following phosphors: YAG \cite{bando1998development}, YGAG \cite{Mueller2002}, 258 \cite{MuellerMach2005}, SLA \cite{Pust2014}, PFS \cite{trigain_spectrum}, QD \cite{lumileds2016qd}\cite{osram2019qd}, SALON \cite{Hoerder2019}.}
\label{fig:phosphor_spectrum}
\end{figure}

\newpage
\subsection{Cost Comparison with Reported Industry Data and DOE Projections}

We conducted a comparison of the LED manufacturing cost structure produced by our cost model with previously reported US DOE calculations and projections based on the LEDCOM model and industry data provided as part of industry round table discussions (see \cref{fig:costmodel_calibration}). We note some differences between the results of our model and the cost structure reported or projected by the DOE. For instance, the share of the epitaxy step is consistently larger in the DOE data. This can in part be explained by our model relying on state-of-the-art equipment at a virtual US-based manufacturing location, while industry might not run low-power and mid-power chip production on these, more expensive, reactors. In addition, the manufacturing lines of the majority of manufacturers is located in Asia. We also note that the share of the substrate price in the DOE data is much larger than in our model, which can in part be explained by the overestimation of the actual price of sapphire wafers in earlier projections for 2015-2020. Finally, the relative importance of the packaging part of the manufacturing process is very similar in our model and in the DOE results in 2012. However, in the DOE projections it significantly decreases by 2020, while in our model it retains and even increases its share of the total cost. This trend has been independently confirmed by researchers and industry reports on wafer-level packaging \cite{Lee2011WPL}\cite{Xie2013}\cite{ledsmag2017WLP}, showing better performance of our cost model compared to earlier DOE model projections

\begin{figure}[h]
	\centering
    \includegraphics[width=\textwidth]{./figures/costmodel_calibration.pdf}
	\caption{Comparison of LED manufacturing cost structure between our manufacturing cost model (bottom panel) and previously published US DOE manufacturing cost calculations and projections based on industry data and LEDCOM model (top panel). Hatched bars are projections given by US DOE for 2018 and 2020. Sources for top panel: \cite{doe2010solid}\cite{doe2011solid}\cite{doe2012solid}\cite{doe2013solid}\cite{doe2014solid}\cite{doe2015solid}\cite{doe2016solid}. Sources for bottom panel: own elaboration based on the cost model described in \cref{sec:costmodel}.}
	\label{fig:costmodel_calibration}
\end{figure}

\clearpage
\section{Complete List of Sources for Figures}
\label{sec:sources}

\subsection{Figure 1 (Historical Development of Luminous Efficacy)}

The list of sources for Figure 1 in the main article is organized by different technologies:

\begin{table}[h!]
    \begin{NiceTabularX}{\textwidth}{|l|X|}
    \hline
    \textit{Technology} & \textit{References} \\
    \hline
    LED & own research, compare \cite{zenodo_weinold_led_history} \\
    \hline
    CFL (<1984) & \cite{Bouwknegt1982}\cite{Vrenken1983} \\
    \hline
    CFL (1984-2011) & \cite{eger2018origin} \\
    \hline
    CFL (>2011) & \cite{Guan2015} \\
    \hline
    Fire, Incandescent, HID & \cite{azevedo2009transition} augmented by own calculations based on \cite{benesch1905beleuchtungswesen} \\
    \hline
    Max. efficacy & \cite{Murphy2012} \\
    \hline
    \end{NiceTabularX}
\end{table}

\subsection{Figure 2 (Historical Development of Lamp Prices)}

The sources for Figure 2 in the main article are grouped by the provider of data:

\begin{table}[h!]
    \begin{NiceTabularX}{\textwidth}{|l|X|}
    \hline
    \textit{Provider of References} & \textit{References} \\
    \hline
    Stiftung Warentest (Germany) & \cite{Warentest2008}\cite{Warentest2009_1}\cite{Warentest2009_2}\cite{Warentest2010_1}\cite{Warentest2010_2}\cite{Warentest2011}\cite{Warentest2012}\cite{Warentest2013}\cite{Warentest2014_1}\cite{Warentest2014_2}\cite{Warentest2015}\cite{Warentest2016_1}\cite{Warentest2016_2}\cite{Warentest2018} \\
    \hline
    Konsument (Austria) & \cite{Konsument2010} \\
    \hline
    Which (UK) & \cite{Which2020} \\
    \hline
    Industry Periodical & \cite{PM2020} \\
    \hline
    Government Report & \cite{council2013assessment} \\
    \hline
    \end{NiceTabularX}
\end{table}

\subsection{Figure 3 (Historical Evolution of LED Chip Architectures)}

The sources for Figure 3 in the main article are grouped by their type:

\begin{table}[h!]
    \begin{NiceTabularX}{\textwidth}{|l|X|}
    \hline
    \textit{Type of Source} & \textit{References} \\
    \hline
    Scientific Publications & \cite{plossl2010wafer}\cite{bierhuizen2007performance}\cite{gencc2019distributed}\cite{chong2014performance} \\
    \hline
    Patents & \cite{patent1999uemura}\cite{patent1998takaoka}\cite{patent1999komaki}\cite{patent1999komaki}\cite{ludowise2006resonant}\cite{camras2005iii}\cite{steigerwald2004contacting} \\
    \hline
    Other Publications & \cite{craford2015}\cite{sun2016}\cite{yole2013packaging} \\
    \hline
    \end{NiceTabularX}
\end{table}

\subsection{\cref{fgr:subeff} (Historical Developments in Device Sub-Efficiencies)}

The sources for \cref{fgr:subeff} are grouped by sub-efficiencies represented on different panels of the figure:

\begin{table}[h!]
    \begin{NiceTabularX}{\textwidth}{|l|X|}
    \hline
    \textit{Panel (Sub-Eff.)} & \textit{References} \\
    \hline
    Panel A1 ($V_f$) & \cite{nichia2001data}\cite{lumi2002data}\cite{gen2005data}\cite{candlepwr2005data}\cite{lumi2006data}\cite{lumi2007data}\cite{nichia2008data}\cite{lumi2008data}\cite{osram2008data}\cite{jeong2011high}\cite{osram2012data}\cite{osram2013data}\cite{osram2014data} \newline \cite{lumi2016data_1}\cite{lumi2016data_2}\cite{epistar2017data}\cite{osram2017data_1}\cite{osram2017data_2}\cite{samsung2017data}\cite{samsung2018data}\cite{osram2018data}\cite{epistar2018data}\cite{lumi2019data} \\
    \hline
    Panel A2 (Droop) & Data calculated from luminous intensity curves of respective device datasheets: \cite{datasheet_osram_topled}\cite{osram2008data}\cite{osram2008gdplus}\cite{osram2018csp}\cite{datasheet_lumileds_lux1}\cite{lumi2008data}\cite{lumi2016data_1}\cite{lumi2016data_2}\cite{samsung2018data} \\
    \hline
    Panel B1 (IQE) & Data extracted from respective device datasheets and scientific publications (for a complete list, compare \cite{zenodo_weinold_led_history}). Additional data was extracted from government reports: \newline
\cite{doe_ssl_multiyear_2006}\cite{doe_ssl_multiyear_2007}\cite{doe_ssl_multiyear_2008}\cite{doe_ssl_multiyear_2009}\cite{doe_ssl_multiyear_2010}\cite{doe_ssl_multiyear_2011}\cite{doe_ssl_multiyear_2012}\cite{doe_ssl_multiyear_2013}\cite{doe_ssl_multiyear_2014}\cite{doe_ssl_rnd_2015}\cite{doe_ssl_rnd_2016} \\
    \hline
    Panel B2 (LEE) & \cite{lee2005analysis}\cite{krames2007status}\cite{Jang2004}\cite{Horng2013}\cite{Liao2010}\cite{HungWenHuang2005}\cite{Leem2007}\cite{Huang2008}\cite{Wang2009}\cite{Huh2003}\cite{Horng2008}\cite{Gao2008}\cite{Chang2003}\cite{Zhou2012} \newline \cite{ChunJuTun2006}\cite{Hua2009}\cite{Matioli2010}
\cite{lee2005analysis}\cite{Zhu2015}\cite{Ding2015}\cite{Taki2019}\cite{Shchekin2006}\cite{Hu2016}\cite{Horng2010}\cite{Lin2016}\cite{Yue2018}\cite{Zhao2012}\cite{Zhu2015}\newline \cite{Ding2015}\cite{wierer2001high}\cite{Steigerwald2002}\cite{DaeSeobHan2006}\cite{Wang2006}\cite{Lee2007}\cite{Shen2007}\cite{Huang2006}\cite{Zhmakin2011} \\
    \hline
    \end{NiceTabularX}
\end{table}

\newpage

\printbibliography

\end{document}