\documentclass[parskip=full]{article}

% functionality
\input{./auxiliary/preamble/1_preamble_functionality_main}

% layout
\input{./auxiliary/preamble/2_preamble_layout_paperdraft}

% bibliography
\usepackage[
    backend=biber,
    style=ieee
]{biblatex}

% show DOI URL (https://doi.org/XXX.XXXXX.XXXX), instead of publisher URL (https://springer.com/XXXX)
% compare https://tex.stackexchange.com/a/616241
\DeclareSourcemap{
  \maps[datatype = bibtex]{
    \map{
      \step[notfield = keywords, final]
      \step[fieldsource = doi, final]
      \step[fieldset = url, null]
    }
    \map{
      \step[fieldsource = keywords, notmatch = \regexp{\bprimary\b}, final]
      \step[fieldsource = doi, final]
      \step[fieldset = url, null]
    }
  }
}
\AtEveryBibitem{
    \clearfield{urlyear}
    \clearfield{urlmonth}
}

% add exact location of equation in source citation (eg. 'cf. [1] (1.123)')
% compare https://tex.stackexchange.com/a/668944
\usepackage{mathtools}
\newtagform{tagcite}{(}{)}
\newcommand*{\tagcite}[2]{%
    \renewtagform{tagcite}{(}{, cf.\@ \cite{#1} (#2))}
    \usetagform{tagcite}
}
\AfterEndEnvironment{equation}{\usetagform{default}}
\addbibresource{bibliography.bib}

% number all lines as pre journal submission guidelines
\usepackage{lineno}
\linenumbers

%% METADATA %%%%%%%%%%%%%%%%%%%%%%%%%%%%%%%%%%%%%%%%%%%%%%%%

\hypersetup{
    pdftitle={Main Article},
    pdfauthor={Michael Philipp Weinold, Sergey Kolesnikov, Laura Diaz Anadon},
}

%% MAIN DOCUMENT %%%%%%%%%%%%%%%%%%%%%%%%%%%%%%%%%%%%%%%%%%%

\begin{document}

\begin{center}
    \large
    \textbf{Rapid technological progress in white light-emitting diodes and its sources in innovation and technology spillovers}
\end{center}

M.P. Weinold$^{1,2,3}$, S. Kolesnikov$^1$, L.D. Anadon$^{1,4}$ \\
\newline
$^1$ Centre for Environment, Energy and Natural Resource Governance Department of Land Economy, University of Cambridge, Cambridge, UK \\
$^2$ Group for Energy Systems Analysis, Department of Mechanical and Process Engineering, ETH Zurich, Zurich, Switzerland \\
$^3$ Technology Assessment Group, Paul Scherrer Institute, Villigen, Switzerland \\
$^4$ Belfer Center for Science and International Affairs, Harvard University, Cambridge MA, USA

\begin{abstract}

Since their introduction to the market in 1996, white light-emitting diodes (LEDs) have greatly improved in performance, efficiency, and manufacturing cost. Understanding the extent and mechanisms of the rapid progress in white LED technology can provide valuable insights for accelerating innovation in other demand-side clean energy technologies critical for reducing global carbon emissions. Here we show, through cost and performance modeling based on data from literature review, patent analysis, and expert interviews, that the efficiency of top-performing warm white GaN-based LED packages increased from 5.8\% in 2003 to 38.8\% in 2020. Over the same period, the manufacturing cost of low-to-mid-power LED packages decreased by 95.5\% from \$1.1 to \$0.05 (in 2020 USD). We find that technology spillovers from other sectors accounted for at least 8.5\% of efficiency improvements and nearly all consumer experience enhancements, playing an important role in widespread LED adoption in lighting.

\end{abstract}

\section*{Keywords}

solid-state lighting, light emitting diodes, innovation, technology spillovers, energy efficiency, cost reductions, consumer experience metrics

\section{Main}
\label{sec:intro}

A rapid reduction of global carbon dioxide emissions is urgently required in order to mitigate the effects of climate change \cite{Forster2019}. While the 2015 Paris Agreement requires signatory states to develop national emissions reduction strategies, these are not legally binding \cite{bodansky2016legal}. However, some countries have introduced more ambitious legislation, including the European Union, which has set a target of net zero emissions by 2050. It aims to meet this goal through an ambitious set of policies known as the European Green Deal \cite{eu2020green}.

Achieving such ambitious and critically important targets requires both developing new clean energy technologies \cite{iea2020cleanenergy} and accelerating the deployment of existing supply-side \cite{sinn2012green} and demand-side energy technologies \cite{rgeVorsatz2009}. To ensure rapid adoption of these technologies, significant reductions in their costs and improvements in their performance are needed. This requires understanding how these cost reductions and performance improvements can be achieved \cite{nemet2019solar,Stephan2021,kavlak2018evaluating,Ziegler2021}.

Improvements in technology cost and performance, which in this study we generally call “innovations”, can occur via different mechanisms, including (but not limited to) learning by doing \cite{WRIGHT_1936, Arrow_1962}, targeted research and development (R\&D) efforts, or economies of scale \cite{johansson2012global, iea2020perspectives}. The role of these mechanisms at different stages of the technology life cycle, from research and development to demonstration, market formation and diffusion \cite{grubler2012policies}, is an area of active research \cite{kavlak2018evaluating,Ziegler2021}. Different mechanisms can result in different kinds of innovations, characterized by what changes in the technology (e.g., its architecture, materials, components, or manufacturing process) and the scale of resulting improvements (e.g., radical or incremental) \cite{Acemoglu_2022}. For example, R\&D efforts (sometimes also called “learning-by-researching”) \cite{Castelnuovo_2005} more often than other mechanisms can result in substantial or even radical improvements in technology performance. On the other hand, learning-by-doing is typically associated with continuous incremental improvements in the technology resulting from accumulated experience in technology demonstration, manufacturing, and use. Both mechanisms, as well as economies of scale, can also lead to radical (i.e., discontinuous) or incremental reductions in technology costs. 

Importantly, different mechanisms of innovation in a particular technology play out within a broader innovation system comprising multiple actors, networks, and institutions \cite{grubler2012policies, Anadon2016}. As a result, these mechanisms are affected by various drivers and barriers, including forces of supply and demand \cite{Mowery1979}, public policies, and regulation \cite{anadon2009policy}.

Among these drivers, the role of external knowledge and technology spillovers in research and development of energy technologies remains insufficiently studied \cite{Stephan2021}. While the exact definition of spillovers in the literature depends on the context \cite{Nemet2012, kolesnikov2021spillovers}, we specifically follow the approach of two previous studies \cite{Stephan2021, kolesnikov2021spillovers}. We define technology spillovers as the application of external knowledge in a technology where knowledge is considered external if it has been developed for application in other technologies, sectors, or scientific disciplines. There is emerging evidence that understanding spillovers and the knowledge network beyond a particular technology may be an important factor in understanding \cite{Pichler2020, iea2020cleanenergy} and shaping \cite{Clark2016, Stephan2021} the future evolution of technologies. For example, spillovers can provide critical external knowledge inputs to the R\&D process \cite{kolesnikov2021spillovers}, enable the re-use of experience from different sectors in learning-by-doing \cite{iea2020cleanenergy}, and lead to cost reductions, e.g., via repurposing the manufacturing equipment from other sectors \cite{Stephan2021}.

The aim of this paper is to identify and, where possible, quantify the contribution of different mechanisms and technology spillovers to the historical progress in a specific technology – white light-emitting diodes (LEDs) for general illumination. Among demand-side technologies, lighting is a particularly important area for climate change mitigation efforts, as it currently accounts for 15-19\% of global electricity consumption \cite{Zissis2016,doe_electricity}. It is also an area of rapid recent technological change: since the introduction of first commercial white LEDs in 1996 \cite{Nakamura2013}, lighting technology has experienced dramatic efficiency improvements. As shown in \cref{fgr:history_efficacy}, thanks to the introduction of LED-based solid-state lighting (SSL), the efficacy of lighting sources has increased by three orders of magnitude in just over 20 years, which is significantly faster than the historical progress observed in previous lighting technologies \cite{weinold2021quantifying}. For comparison, the highest performing light-emitting devices today reach efficacies of 220 lm/W \cite{lumistrips2021mid}, while an incandescent light bulb can only reach efficacies of up to 18 lm/W. Moreover, this rapid improvement in efficiency has been accompanied by a similarly impressive decrease in LED manufacturing costs and retail prices. \cref{fgr:cost_lamp_small} shows how LED retail prices have fallen by two orders of magnitude, at an annual price per flux decline of 27.3\% during the 2008-2020 period, in line with previous estimates \cite{Gerke2020}.

\begin{figure}[h!]
 \centering
 \includegraphics[width=\textwidth]{figures/history_efficacy.pdf}
 \caption{\textbf{Historical progress in the luminous efficacy ($\eta$) of the most widely-used lighting technologies in human history.} Data points indicate best performers by year of market introduction. Luminous efficacy is the measure of how efficiently a light source converts electrical energy into visible light that can be perceived by the human eye, taking into account the wavelength sensitivity of the eye (see Supplementary Note 2 for details). Dashed lines are guides to the eye, based on a 3rd-order polynomial fit to the data trend. The physical limit for an ideal light source with a colour rendering index of CRI=90, denoted as $K_{max}^{CRI90}$, is shown as a black horizontal line, as per calculations by Murphy et al. \cite{Murphy2012}. The magnified plot shows the progress in cool white LEDs from 1996 to 2020, with the dashed line indicating a linear rate of efficacy improvement of 10lm/W per year. For comparison, efficacies of best performers in legacy lighting technologies for 2020 are shown as coloured horizontal lines. Note the logarithmic scale of the vertical axis on the main plot and the linear scale on the magnified plot. Abbreviations: HID - High-Intensity Discharge; CFL - Compact Fluorescent Lamp; Hal. - Halogen, Incd. - Incandescent. Source: own synthesis of published data based on a visual approach proposed by Azevedo et al. \cite{azevedo2009transition}. See Supplementary Note 8 for the full list of sources and references.}
 \label{fgr:history_efficacy}
\end{figure}

\begin{figure}[h!]
\centering
  \includegraphics[height=5cm]{figures/cost_lamp_small.pdf}
  \caption{\textbf{Historical development of retail sales prices per luminous flux for LED-based luminaires in the period 2008-2020.} Luminaires considered include light bulbs, spotlights and recessed lights. Red curved and dashed lines represent average retail sales prices and price projections for LED based luminaires published by the U.S. Department of Energy (DOE) \cite{national2013assessment}. Shown for reference are the average prices for compact fluorescent (CFL) and incandescent light bulbs, with the latter assumed constant based on the average in the covered time period. Note that projections are consistently at the lower end of the retail price data. This is because retail price data includes more expensive luminaire types, while projections are for traditional light bulbs only. Projections also do not include value added tax. Source: own synthesis of data on sales prices collected from various consumer watchdog databases and publications. See Supplementary Note 8 for the full list of sources and references.}
  \label{fgr:cost_lamp_small}
\end{figure}

These dramatic improvements in lighting technology, supported by the introduction of efficiency regulations phasing out incandescent light bulbs and targeted policies stimulating LED adoption in many countries, led to the rapid expansion and diffusion of SSL technologies \cite{weinold2020long, stegmaier2021incandescent, Mills2014}. As a result, by 2020, highly efficient LED luminaires were saving an estimated 131 TWh/year in the EU \cite{eu2019impactass} and 442 TWh/year in the US \cite{guidehouse2020adoption}, which is on par with the amount of energy produced in the same year by all photovoltaic installations in these regions. Notably, market adoption of LED lighting is not limited to developed economies \cite{Kamat2020}. For example, durability, low up-front cost and high efficiency of LED light sources have led to early and widespread adoption in rural West African communities without access to grid electricity \cite{Bensch2017}. LEDs have also been used in a wide range of applications beyond lighting, such as personal health monitors \cite{Wyatt2020}, potable water treatment \cite{Lui2014}, high-bandwidth wireless data transmission \cite{Haas2016}, and augmented reality eye wear \cite{Lee2016}. 

Despite this impressive history, the mechanisms of technological progress in white LEDs have not received as much attention from researchers as in supply-side energy technologies, such as solar photovoltaics \cite{kavlak2018evaluating, nemet2019solar}, wind energy \cite{qiu2012price, jennings2020policy}, or lithium-ion batteries \cite{Ziegler2021, Stephan2021}. To the best of our knowledge, no study has comprehensively discussed the mechanisms or extent of historical progress across a range of metrics related to LED cost and performance since the introduction of the first commercial white LED products (see Supplementary Notes section 1 for a brief review of previous literature on this topic). This is consistent with previous observations regarding the marginalization of end-use technologies in the analysis of energy innovation for climate change impact mitigation \cite{Wilson2012, Creutzig2018}. Understanding the extent to which individual innovations and technology spillovers contributed to overall improvements in white LED technology cost and performance, by which mechanism these innovations and spillovers occurred, and what factors affected this process will provide valuable lessons for innovation in other demand-side energy technologies and clean energy innovation in general.

To address these questions, we first identify three sets of metrics to quantify historical progress in LED lighting technology: 1) metrics of energy efficiency of LED devices, including the total device efficiency (“lamp efficiency”) and the sub-efficiencies that describe different energy loss channels in a LED device; 2) metrics of lighting quality relevant to consumer experience, specifically the colour rendering index (CRI) and colour temperature (CT); and 3) LED chip manufacturing cost. The rationale for the choice of these metrics, along with their detailed descriptions and definitions, are discussed in Supplementary Method 2. Next, using data from literature review, patent analysis, and expert interviews, we trace the historical improvements in these metrics in warm white GaN-based phosphor-converted LEDs—currently the most widely adopted variant of white LED technology which we thus choose as the focal technological area for our analysis—from their introduction to the lighting market in 2003 to 2020, the year with the most recent data available at the time of data collection.

We find that total LED device efficiency of such LEDs increased from 5.8\% to 38.8\% between 2003 and 2020. A majority of this increase can be attributed to a series of innovations in white LED technology resulting from R\&D efforts, with at least 8.5\% of the cumulative increase attributed specifically to technological innovations driven by technology spillovers. Our interviews also revealed a lesser but still important role of incremental manufacturing process improvements most likely resulting from learning-by-doing. However, substantial further research is needed to quantify the relative contribution of learning-by-doing vis-a-vis technological innovations to this progress. On the other hand, we find that innovations driven by technology spillovers contributed to nearly all consumer experience improvements in white LED lighting technology, thus playing an important role in widespread adoption of SSL in general illumination applications. Finally, our bottom-up LED manufacturing cost model shows a 95.5\% decrease in the cost of producing low-to-mid-power classic-chip-architecture GaN-based phosphor-converted warm-white LEDs from 1.11\$ to 0.05\$ (in 2020 USD) between 2003 and 2020. In contrast with LED performance improvements, where progress has been driven mostly by R\&D efforts, the dramatic decline in LED manufacturing cost has been a product of learning-by-doing resulting in higher yields across manufacturing steps and economies of scale resulting from increases in the sapphire wafer size. We provide the details of our findings for the progress in white LED efficiency, cost, and consumer experience, as well as the role of technology spillovers in this progress, in the remaining sections below. 


\clearpage

\section{Efficiency Improvements}

Using the mixed-methods approach described in Methods, we collected data on the historical progress in individual sub-efficiencies of warm white phosphor-converted GaN-based LED devices, along with white LED innovations that have driven this progress. The results of our data collection efforts are presented in Supplementary Note 4. Using this information, we calculate the overall lamp efficiency $\eta_L$ for the best performing mid/high-power phosphor-converted warm white LED devices in four years: 2003, 2010, 2016 and 2020. The waterfall diagrams of electric power input losses in \cref{fgr:waterfall} show how improvements in individual sub-efficiencies led to improvements in the overall white LED lamp efficiency from $\eta_L=5.8\%$ in 2003 to 12.7\% in 2010, 32.5\% in 2016 and finally to 38.8\% in 2020. As is evident from the figure, no single loss channel dominates in terms of its contribution to the overall efficiency, in line with previous observations \cite{tsao2010solid}. We note, however, that the loss channels with a fixed physical limit on efficiency, e.g., Stokes loss that determines the light conversion efficiency by phosphors, became relatively more dominant in 2016 and 2020 compared to 2003 and 2010.  This is a direct result of large efficiency improvements in upstream sub-efficiencies.

\cref{fgr:breakthroughs_efficiency} shows the overall magnitude of contributions of identified LED innovations and technology spillovers to improvements in LED efficiency over time across different sub-efficiencies. The full list of identified LED innovations considered in our study is provided in Supplementary Note 5. The list of corresponding technology spillovers is provided in \cref{tab:spillovers}. Through the index decomposition analysis described in Section 2.5 in the Supplementary Information we find that out of the overall LED efficiency increase of 32.9\% from 5.8\% to 38.8\% between 2003 and 2020, at least 2.8 percentage points can be attributed specifically to innovations driven by technology spillovers identified in this study, corresponding to 8.5\% of the total LED efficiency improvements between 2003 and 2020.

In \cref{fgr:breakthroughs_efficiency} we also compare, for the first time, efficiency improvements across sub-efficiencies over time, contrasting them with the physical limits of the corresponding loss channels. Since we focus on a best case scenario, and because we lack reliable information about the distribution of input parameters, which would require us to arbitrarily assign range values for inputs, these estimates do not show uncertainty ranges. Thus, our point estimates should be interpreted as the expected values in a best case scenario. We find that there has been consistent progress across all device sub-efficiencies in the recorded period. Specifically, between 2003 and 2020, forward voltage efficiency increased from 70\% to 99.5\%, internal quantum efficiency from 55\% to 90\%, electrical droop from 65\% to 90\%, light extraction efficiency from 60\% to 90\%, spectral efficiency from 74\% to 83\%, conversion efficiency (red) from 11\% to 45\%, conversion efficiency (green) from 19\% to 61\%. Notably, some sub-efficiencies for the most recent devices considered in our study are now within $\sim10\%$ of their respective physical limits. The exception is spectral efficiency which, at $\sim17\%$ below the physical limit, shows larger potential for further improvements, which is important, given that efforts to improve white LED performance across different efficiency loss channels through R\&D continue \cite{cho2017white, Weisbuch2020}. 

\begin{figure}[H]
 \centering
 \includegraphics[width=15.5cm]{figures/waterfall_performance_2003.pdf}
 \includegraphics[width=15.5cm]{figures/waterfall_performance_2010.pdf}
 \includegraphics[width=15.5cm]{figures/waterfall_performance_2016.pdf}
 \includegraphics[width=15.5cm]{figures/waterfall_performance_2020.pdf}
 \captionsetup{font=footnotesize}
 \caption{\textbf{Waterfall diagrams of the loss channels in a generic mid/high-power white LED package for 2003, 2010, 2016 and 2020}. Losses are normalized to 1W of electric power input. The sub-efficiencies corresponding to each loss channel are listed below each column and described in Supplementary Note 2. Numbers for each loss channel indicate energy losses both in relative terms of input power (in percent) at the point of the channel and absolute values (in Watts). Percentages for loss channels labeled by red, green and blue font indicate losses of corresponding remaining red/green/blue light energy. The following LED architectures and light-source spectra used in calculations for each considered year are shown for reference: 2003 – Flip-chip with YGAG phosphor; 2010 – Flip-chip with 258 phosphor; 2016 – Flip-chip with PFS phosphor; 2020 – Chip-scale package flip-chip with SALON phosphor. Note that the inset spectra clearly show progress in spectral efficiency. The red curve shows the wavelength-dependent sensitivity of the human eye, while the black curve shows the spectrum of the light source. When the spectrum has significant energy at the edges of the sensitivity curve, energy is lost. Details for each architecture are provided in Supplementary Note 3.  Details on the phosphors for light down conversion are provided in \cref{tab:phosphors} and Supplementary Note 4. Abbreviations: Scat. = Scattering; Nonrad. = Nonradiative. Source: own elaboration based on data on sub-efficiencies presented in Supplementary Figure 9 and spectral data in Supplementary Figure 10, adapted from publications on the respective phosphors: YGAG (2003)\cite{Mueller2002}, 258 (2010)\cite{MuellerMach2005}, PFS (2016)\cite{Murphy2015}, SALON (2019)\cite{Hoerder2019}.}
 \label{fgr:waterfall}
\end{figure}

\begin{figure}[H]
 \centering
 \includegraphics[width=\textwidth]{figures/breakthroughs_efficiency.pdf}
 \caption{\textbf{Contribution of innovations and technology spillovers to the historical progress in sub-efficiencies of phosphor-converted warm white LEDs.} LEDs with test currents of at least 350mA are considered. The following sub-efficiencies are represented: $\eta_{V_f}$ - forward voltage efficiency; $\eta_IQ$ - internal quantum efficiency; $\eta_{droop}$ - efficiency droop; $\eta_{LE}$ - light extraction efficiency; $\eta_{C_R}$ - conversion efficiency for red phosphors; $\eta_{C_{Y/G}}$ - conversion efficiency for yellow/green phosphors; $\eta_S$ - spectral efficiency. The overall LED lamp efficiency $\eta_L$ is displayed in the rightmost column. Vertical bars represent cumulative contributions to efficiency improvements from LED technology innovations, with purple bars indicating innovations driven by technology spillovers (annotated and listed in an inset table) and grey bars indicating all other improvements identified in this study. Horizontal coloured lines indicate sub-efficiency levels of best performing LEDs for the four years used in \cref{fgr:waterfall}: 2003, 2010, 2016, and 2020. For additional historical context, data for 1997 is included whenever possible. "N/A" denotes sub-efficiencies where 1997 data could not be calculated for the following reasons: $\eta_{V_f}, \eta_{droop}$ depend on the device current, which was below 350mA in 1997, making a comparison with contemporary devices difficult. $\eta_{C_R}, \eta_{C_{Y/G}}$ and $\eta_S$ are relevant only to warm white spectrum LEDs, which were not available in 1997. Note: ITO current spreading layer affects different sub-efficiencies in different chip architectures, e.g., in modern flip-chip architectures $\eta_{LE}$ no longer depends on ITO, see Supplementary Note 3. Physical limits on sub-efficiencies are indicated by black horizontal lines. Note that the physical limit of $\eta_{V_f}$ is above 100\% due to quantum effects \cite{david2016electrical}. Source: own elaboration based on data represented in \cref{fgr:waterfall}, \cref{tab:spillovers} and Supplementary Note 4.}
 \label{fgr:breakthroughs_efficiency}
\end{figure}

\section{Consumer Experience Improvements}

Historical improvements in consumer experience metrics for phosphor-converted warm white GaN-based LEDs are shown in \cref{fgr:consumer_experience}. In general illumination applications, a high colour rendering index (CRI) in combination with a specific, tunable range of possible colour temperatures is desirable. Both metrics are determined by LED device emission spectra, which, in turn, depend on the properties of materials used for the conversion of blue light generated by conventional GaN-based LEDs into white light. It allows us to establish the links between all major improvements in the two consumer experience metrics considered in this study and individual LED innovations  associated either with phosphors or quantum dots used for light down conversion in LEDs. The first commercial white LED produced by Nichia in 1996 used a YAG (Yttrium Aluminium Garnet) phosphor that resulted in cool white light only \cite{bando1998development}. As shown in \cref{fgr:consumer_experience}, after a series of innovations listed in \cref{tab:phosphors}, LEDs today can be tuned for high CRI performance and a range of desirable colour temperatures. All innovations in this list are primarily a result of R\&D efforts.

Notably, from detailed descriptions of the history of innovations in this list, provided in Supplementary Note 6, we find that only a single innovation related to LED consumer experience improvements was purposefully developed for application in solid-state lighting: the 2016 SALON phosphor compound \cite{Hoerder2019,seibald2019phosphor}. All other innovations in the list were either originally developed for non-LED applications or prominently used knowledge from areas of science and technology beyond LED or SSL (see also \cref{tab:spillovers}). This means that technology spillovers contributed to nearly all consumer experience improvements in LED-based lighting technology, thus playing an important role in widespread adoption of SSL in general illumination applications. 

\begin{table}[h!]
    \footnotesize
    \centering
    \caption{\textbf{LED innovations related to improvements in consumer experience metrics.}}
    \begin{NiceTabularX}{\textwidth}{|l|l|l|X|X|}
    \hline
        \textbf{Year} & \textbf{Desig.} & \textbf{Chemical formula} & \textbf{Description} & \textbf{Significance} \\ \hline
        1996 & YAG & $Y_3 Al_5 O_{12}:Ce$ & Yttrium aluminium garnet (YAG) phosphor activated with cerium & First LED phosphor, enabled white LEDs \\ \hline
        1996 & YGAG & $(Y_{1-x} Gd_x)_3 Al_5 O_{12}:Ce$ & Gadolinium-doped YAG phosphor & First red-shifted phosphor, enabled warm white LEDs \\ \hline
        2002 & 258 & $(Ba,Sr)_2 Si_5 N_8:Eu^{2+}$ & Europium-doped nitridosilicate phosphor & First red LED phosphor \\ \hline
        2003 & QD & N/A & Quantum dot-based phosphor & First use of QD for LED light down conversion \\ \hline
        2005 & PFS & $K_2 SiF_6: Mn^{4+}$ & Manganese-activated potassium fluorosilicate (PFS) phosphor & First ultra-narrow-band red LED phosphor \\ \hline
        2013 & SLA & $Sr[Li Al]_3 N_4 ]:Eu^{2+}$ & Europium-doped cuboidal nitridolithoaluminate phosphor & Improved narrow-band red phosphor \\ \hline
        2016 & SALON & $Sr[Li_2 Al_2 O_2 N_2]:Eu^{2+}$ & Europium-doped oxonitride phosphor & High-performance ultra-narrow-band red phosphor\\ \hline
    \end{NiceTabularX}
    \caption*{Note: \textit{Year} column represents the earliest reported application of innovation in white LEDs. These differ from the years shown in \cref{fgr:consumer_experience} which correspond to the earliest publication of spectral data for LED products that relied on those innovations. Detailed descriptions of the history of each innovation and relevant spectral data are provided in Supplementary Note 6. Desig.=Designation, Ref.=Reference.}
    \label{tab:phosphors}
\end{table}

\begin{figure}[h!]
 \centering
 \vspace{-5mm}
 \includegraphics[width=\textwidth]{figures/breakthroughs_consumer-experience.pdf}
 \caption{\textbf{Historical improvements in consumer experience metrics of phosphor-converted white LEDs.} Data points show the color temperature (CT) and colour rendering index (CRI) performance of the earliest identified representative white LED products with published spectral data that used phosphors listed in \cref{tab:phosphors}, each indicated by the phosphor label and publication year. Corresponding spectral data used to calculate CT values is provided in Supplementary Note 4. Three data points representing compact fluorescent lamps (CFLs)\cite{cie_reference}, shown as black triangles, are provided for comparison. Horizontal lines represent typical CTs of “traditional” light sources, shown for reference. The desirable range of colour temperatures for home illumination, indicated by a vertical red arrow, lies between two horizontal orange lines representing typical incandescent light and warm daylight colour temperatures. Horizontal red arrow indicates desirable higher values of colour rendering index (CRI). See Supplementary Note 8 for the full list of sources and references of all phosphor represented.}
 \label{fgr:consumer_experience}
\end{figure}

\clearpage
\section{Manufacturing Cost Improvements}

\cref{fgr:costmodel} shows main results of our manufacturing cost modelling for low-to-mid-power classic chip GaN phosphor-converted white LED packages. We find that the manufacturing cost of a single such LED decreased from 1.11\$ (in 2020 USD) in 2003 to 0.11\$ (2020 USD) in 2012 and 0.05\$ in 2020, a 95.5\% overall decrease. Our model shows that for the wafer-level manufacturing steps illustrated in Supplementary Figures 1-5, improved manufacturing yields and increases in the wafer size are jointly responsible for the largest contribution — 95.5\% — to the overall reduction of manufacturing cost per LED chip between 2003 and 2020.

In the case of manufacturing yield, the higher it is, the less inputs are wasted on the production of a single LED package. With the total manufacturing yield dramatically improving from $\sim25\%$ in 2003 to $\sim75\%$ in 2020 (compare \cref{fgr:costmodel}, Panel D), it is not surprising that the total yielded LED manufacturing cost significantly declined over this period.

In the case of wafer diameter, the larger the wafer, the more LED chips can be produced from a single wafer. The wafer diameter commonly used in LED manufacturing has been steadily increasing from 50mm ("2 inch"\footnote{Industry measures wafer size in mm. Still, a 50mm=1.9685Inch wafer is frequently referred to as a "2-inch wafer".}) in 2003 to 200mm (“8 inch") in 2020. We capture this effect in the model by calculating the associated number of die per wafer (DPW) for each representative wafer diameter d \cite{de2005investigation}: $d(2003)\sim 50 mm \rightarrow851 DPW$, $d(2020)\sim200 mm \rightarrow 26,838 DPW$ (see Supplementary Figure 7 in Supplementary Note 3). With more than a thirty-fold increase in the number of die per wafer between 2003 and 2020, the contribution of the whole-of-wafer processing steps to the total cost of manufacturing an individual LED chip and package has dramatically declined over time. As the number of die per wafer increases, the packaging steps, which in the classic chip architecture must be performed separately for each individual LED chip, carry a significantly larger share of the total cost in 2020 than in 2003 (compare Panels A-C in \cref{fgr:costmodel}). However, as our interviewees noted, while growing LEDs on larger wafers is economically desirable, it is associated with engineering and epitaxy challenges as well as high up-front cost of new equipment.

Surprisingly, R\&D efforts to improve LED technology and associated technology spillovers are not found to have a substantial impact on LED manufacturing cost reductions over time. There are several explanations for this finding. First, the functional unit of our study is the LED chip itself, not the luminaire containing multiple chips. As LED efficiency and overall brightness have increased, a smaller number of chips can be used in luminaires to obtain the same level of the overall light flux. However, since we are investigating improvements at the level of single chips, we do not capture the cost effects downstream of chip manufacturing. We also previously showed that overall brightness is no longer a target metric in chip development \cite{weinold2021compound}. Second, we find that an increase in wafer size and higher overall yield, taken together, are jointly responsible for a 95.5\% reduction of manufacturing cost per LED chip between 2003 and 2020. Both were enabled by advances in manufacturing equipment (e.g., in epitaxy) enabling the economies of scale in manufacturing, as well as incremental manufacturing process improvements from learning-by-doing, rather than LED technology innovations stemming from R\&D, which was confirmed by our interviews with experts in LED manufacturing.

Notably, our bottom-up cost model is constructed to provide process-step resolution across three different key chip architectures: classical chips, flip chips, and chip-scale package flip chips. However, in this study we were able to collect data and compare the outcomes only for the classical chip architecture. Collecting the full set of data needed to populate the model for the remaining two architectures requires access to proprietary information from industry. With this limitation, tracking manufacturing cost reductions across all three key LED chip architectures remains a topic for future work.

Our findings are further supported by a preliminary sensitivity analysis, presented in Supplementary Discussion 1, where we find that the sensitivity of the cost model to variation in its main parameters decreases over time with the increase of the number of DPW. 

In Supplementary Discussion 2, we also provide a comparison of our model results with past cost report and projections published by the U.S. Department of Energy (DOE) on the basis of the \textit{LEDCOM} cost model \cite{ledcomv2} and industry data reported as part of SSL round table meetings.

\begin{figure}[ht!]
\centering
\includegraphics[width=17.5cm]{figures/costmodel_results_years.pdf}
\caption{\textbf{Historical changes in white LED manufacturing cost structure.} Presented cost calculations are an outcome of our cost model (see \cref{sec:methods} and Supplementary Method 6 for a single low-to-mid power GaN-based, classic chip, phosphor-converted white LED package, assuming an ideal factory with state-of-the-art equipment at a U.S. location. Panels A-C: Waterfall diagrams of white LED manufacturing cost structure split by manufacturing process steps for years 2003, 2012 and 2020. Process steps on the horizontal axis are sequenced from left to right in the same order as in the modelled LED manufacturing process. Panel D: Cumulative manufacturing yield after each process step for years 2003, 2012 and 2020. For a graphic representation of the manufacturing process, see the diagrams in Supplementary Note 2. Abbreviations: Litho – Lithographic Process, Insp. – Inspection, Depn. – Deposition, CMP - Chemical-Mechanical Planarization.}
\label{fgr:costmodel}
\end{figure}

\clearpage
\section{Technology Spillovers}

Following the frameworks for the analysis of technology spillovers proposed by Stephan et al. \cite{Stephan2021} and Kolesnikov et al. \cite{kolesnikov2022technology}, we synthesize information from the interviews, historical records, literature review and citations in patents and publications to identify which innovations in white LED technology were driven by external knowledge originating in areas of science and technology beyond white LED lighting. We find that at least nine LED innovations, listed in \cref{tab:spillovers}, clearly involved such external knowledge. We then analyze the sources, mechanisms and enablers of corresponding technology spillovers, as well as their contributions to progress in white LED technology. 

We find that three spillovers associated with the use of YAG/YGAG phosphors in LEDs played the key role in the development of the first commercial white LED lighting products, essentially enabling the solid-state lighting market and industry of today. As noted above, subsequent spillovers then contributed to white LED technology innovations that were cumulatively responsible for at least 8.5\% of the total LED lamp efficiency improvements between 2003 and 2020, as well as nearly all key improvements in consumer experience metrics.

Among the spillover sources, we find that all nine spillovers had origins in the scientific disciplines such as various branches of chemistry, materials science, optics and photonics, and solid-state physics. Five spillovers also utilized technical knowledge and expertise in cathode ray tubes, fluorescent lighting, optoelectronic devices, nanotechnology, and nature-inspired material design.
Among the spillover mechanisms, six spillovers (involved in all phosphors except PFS and SLA, plus ITO) were a result of application of external scientific and technical knowledge already available to researchers and inventors. Three remaining spillovers (involved in PFS and SLA phosphors, plus PSS) occurred as an outcome of targeted search for relevant external knowledge outside the LED domain. In addition, at least two spillovers (involved in 258 and PFS phosphors) occurred as a result of knowledge exchange in direct R\&D collaboration.

Among the enabling factors that we identified for these spillovers, we highlight public mission-driven R\&D funding; industry-academia partnerships; firm experience in multiple industries; conferences that brought together researchers from academia and industry; cultural and language proximity; freedom of search in academia; and university alumni networks.
Finally, we find that, on average, it took 26 years from the initial invention or discovery taking place in a field of science and technology different from white LED lighting to the first time it got applied in (‘spilled over to’) the white LED lighting domain, with this time varying from 5 to almost 70 years. In contrast, it took much less time – just 6 years on average, varying from just a few months to 19 years – to develop a commercial application for the spillover knowledge in the LED lighting market.

\clearpage
\begin{table}[h!]
    \tiny
    \centering
    \caption{\textbf{Technology spillovers involved in white LED technology innovations identified in this study.}}
    \begin{NiceTabularX}{1.1\textwidth}{|l|l|l|X|X|X|X|X|X|}
    \hline
        \textbf{Inv.} & \textbf{S/O} & \textbf{Comm.} & \textbf{LED Innovation} & \textbf{Spillover} & \textbf{Enabler} & \textbf{Origin} & \textbf{Ref.} & \textbf{Area of Improvement} \\ \hline
        1926 & 1994 & 1996 & LED phosphors &  Use of phosphors for light down conversion in LEDs & Firm history in phosphor manufacturing for CRT applications & Materials science (S), Cathode ray tubes (T) & \cite{bright1972electric,shimizu1994sheet,cho2017white} & Enabled light down conversion in LEDs \\ \hline
        1967 & 1996 & 1996 & YAG:Ce phosphor & Use of YAG:Ce phosphor in a first white LED product & Firm history in phosphor manufacturing; firm working in multiple sectors & Chemistry (S), Materials science (S), Fluorescent lighting (T), Cathode ray tubes (T) & \cite{blasse1967new,bando1996,bando1998development,shimizu1999light,cho2017white} & Enabled white LED products, $\eta_S$, $\eta_C$ \\ \hline
        1967 & 1996 & $<$2002 & YGAG phosphor & Use of YGAG phosphor in first warm white LEDs & Firm history in phosphor manufacturing; firm working in multiple sectors &  Chemistry (S), Materials science (S) & \cite{holloway1969optical,bando1998development,shimizu1999light,Mueller2002} & Enabled warm white LEDs, $\eta_S$, $\eta_C$ \\ \hline
        1982 & 1996 & $<$2010 & Patterned sapphire substrate (PSS) & Use of anti-reflective properties of substrate patterns in LEDs & not identified & Optics and photonics (S), Materials science and technology (S,T), Nature-inspired material design (T) & \cite{moharam1982diffraction,krames1998ordered,feezell2018invention,Narukawa_2010} & $\eta_{LE}$, $\eta_{IQ}$ (depending on the chip architecture, compare Supplementary Note 3)\\ \hline
        1971 & 1999 & $<$2005 & Indium tin oxide (ITO) current spreading layer & Use of ITO current spreading layer in white LEDs & Public mission-driven R\&D funding; Flexibility of public funding (through DARPA); Industry-academia partnership & Optics and photonics (S), Materials science and technology (S,T), Optoelectronic devices (T) & \cite{vossen1971rf,fraser1972highly,margalith1999indium} & $\eta_{Vf}$, $\eta_{LE}$ (depending on the chip architecture, compare Supplementary Note 3) \\ \hline
        1997 & 2002 & 2005 & 258 phosphor & Use of luminescent ‘258’ nitridosilicate compound as LED phosphor & Freedom of search; Communication at a conference; Cultural and language proximity; International industry-academia partnership & Chemistry (S), Materials science (S) &\cite{Huppertz1997,mueller2004phosphor,MuellerMach2005} & $\eta_S$, $\eta_C$ \\ \hline
        1984 & 2003 & 2009 & Quantum dot-based phosphor & Use of quantum dots for light down conversion in LEDs & Public mission-driven R\&D funding; Communication at a conference - integration enabler; Commercial success on a different market - integration enabler & Solid-state physics (S), Photochemistry (S), Nanotechnology (T) &\cite{fojtik1984photo,simmonsfinal,ledprof_nexxusqd,bourzac2013quantum} & $\eta_S$, $\eta_C$ \\ \hline
        1972 & 2005 & 2015 & PFS phosphor & Use of knowledge in luminescent materials and skills in "wet" chemical synthesis to synthesize PFS compound and optimize it as LED phosphor & Public R\&D funding - integration enabler; International industry-academia partnership; Alumni connections (?) & Chemistry (S), Materials science (S) &\cite{paulusz1973efficient,radkov2009red,Murphy2015} & $\eta_S$, $\eta_C$ \\ \hline
        2008 & 2013 & 2015 & SLA phosphor & Use of knowledge about existing cuboidal nitride compounds to identify and synthesize structurally similar SLA phosphor & Industry-academia partnership & Structural chemistry (S), Materials science (S), Solid-state physics (S) &\cite{Park2008New,schmidt2013new,Pust2014} & $\eta_S$, $\eta_C$ \\ \hline
    \end{NiceTabularX}
    \caption*{Note: Inv. - Year of initial invention, identified by the earliest literature source describing the original invention or idea in a field outside white LED lighting that eventually ‘spilled over’ to the latter field. S/O - Year of spillover to LED; Comm. - Year of commercial application, identified as the year of the first recorded application of that idea or invention in a commercial LED product. Ref. - References. LED innovations are ordered by the year in which a technology spillover into LED occurred, provided in the S/O column. Origin column represents knowledge domains in which spillovers initially emerged, where (S) denotes a scientific discipline and (T) is an area of technology. Ref. column lists literature sources for the represented discoveries, innovations and spillovers. Area of Improvement column represents the impact of spillovers on different aspects of white LED technology, e.g., improvements in particular sub-efficiencies.}
    \label{tab:spillovers}
\end{table}

\clearpage
\section{Discussion}
\label{sec:discussion}

In this study, we use a multi-method approach to synthesize evidence from multiple sources and, for the first time, reconstruct a detailed picture of the rapid technological progress in white LED technology throughout its history since the introduction to the market in 1996 and across an ensemble of metrics related to device performance (\cref{fgr:waterfall} and \cref{fgr:breakthroughs_efficiency}), manufacturing cost (\cref{fgr:costmodel}), and consumer experience (\cref{fgr:consumer_experience}). Improvements in these metrics are traced (with contributions quantified, where possible) to specific white LED technology innovations resulting from R\&D activities, technology spillovers, learning-by-doing, and economies of scale in the manufacturing process. 

Specifically, we find that progress in LED efficiency and consumer experience has been predominantly driven by innovations resulting from R\&D activities, but the relative contribution of technology spillovers to these innovations has been very different: at least 8.5\% for efficiency improvements and virtually all 100\% for consumer experience metrics. In addition, we observe (but so far have not been able to quantify) a likely small but meaningful role of learning-by-doing in efficiency improvements. Interestingly, we find that technology spillovers are most prevalent in those device loss channels that are well understood at the physical level (e.g., light extraction and spectral efficiencies, see \cref{fgr:breakthroughs_efficiency} and \cref{tab:spillovers}). We find that improvements in device loss channels that are governed by complex quantum effects at the atomic level (e.g., droop) and at present can only be described heuristically, have, on the other hand, come mostly from incremental manufacturing process improvements resulting from learning-by-doing with little evidence for spillovers.

In contrast with efficiency and consumer experience improvements, we find that dramatic manufacturing cost reductions in white LEDs have been mostly driven by learning-by-doing allowing higher yields across manufacturing steps, and economies of scale resulting from increases in the sapphire wafer size. We estimate a combined contribution of these two factors to cost reductions for warm white GaN-based phosphor-converted LEDs to be at least 95.5\%, leaving only a minor residual of 4.5\% for all other mechanisms of innovation, including R\&D. This effect of manufacturing scale and learning-by-doing on cost reductions cannot be explained without the demand-pull policies related to regulations and bans around incandescent lightbulbs, which paved the way for scale up and industry investments in manufacturing. In this way, the role of demand-pull policies is pivotal when it comes to cost metrics, as it has been with solar PV \cite{nemet2019solar}, although perhaps even greater with white LED lighting. 

Interestingly, comparable analyses of historical cost reductions for supply-side energy technologies show a very different pattern of relative contributions of different mechanism to cost dynamics. For example, Kavlak et al.\cite{kavlak2018evaluating} showed that for solar photovoltaic modules, increases in wafer size and manufacturing yield (with the latter associated with learning-by-doing) contributed only 11\% and 7\% cost reductions from 1980 to 2012, correspondingly, while innovations resulting from R\&D that increased the efficiency of solar modules reduced the costs by 23\%, with remaining cost declines explained by economies of scale and declines in material input prices. Similarly, Ziegler et al.\cite{Ziegler2021} showed that technological innovations resulting from public and private R\&D accounted for as much as 54\% of cost declines in lithium-ion batteries from the late 1990s to early 2010s. while economies of scale contributed 30\%, and learning-by-doing associated primarily with improved manufacturing yields reduced the costs only by 2\%. Such dramatic differences in relative contributions of the same mechanisms to the cost dynamics between white LED lighting, solar photovoltaics, and lithium-ion batters, as well as our findings on the progress in consumer experience metrics, indicate that LED lighting as a demand-side technology can be more directly affected by end-user needs and demands than technologies focused on energy generation and storage. It may also have sufficiently different characteristics affecting its manufacturability and marketability that require different types of policy support and response than supply-side energy technologies. Whether our findings for white LED lighting are generalizable to broader demand-side technologies, and what are the reasons behind the observed differences in the patterns of technological progress in different types of energy technologies, remains to be explored in future research. 



Our analysis of the sources, mechanisms and enablers of the identified technology spillovers highlights the critical role played by a deep understanding of the physical, chemical and optical phenomena underlying the operation of LEDs, as well as materials science and technology and nanotechnology involved in the production of LEDs, for past and future advances in LED and solid-state lighting technology. Specifically, deep physical understanding of LED device energy loss channels had enabled important innovations in LEDs that increased several sub-efficiencies in LEDs and will continue to do so, as expected by eminent experts in the field whom we interviewed. A practical implication of this finding is that additional research in these areas and a more deliberate search for relevant external knowledge may accelerate expected future advances in LED technology and its applications both in SSL and technology areas beyond general lighting. According to our observations, corresponding knowledge spillovers can be deliberately stimulated, among other factors, by measures such as knowledge exchange events and long-term partnerships between academia and industry, interdisciplinary training and hiring, dedicated mission-driven public R\&D funding, and ensuring certain freedom of search in academia. These observations also further reinforce broader arguments made against the dichotomy of basic and applied research \cite{narayanamurti2016cycles, narayanamurti2021genesis} and the calls for open, inclusive and flexible research cultures \cite{Stephan2021}.

While additional work is necessary to determine if the patterns we find for innovation in white LEDs apply in other demand side technologies, there are some emerging findings that would have implications. First, the fact that efficiency channels with greater physical understanding were more likely to be shaped by spillovers could indicate that investing into research that develops such physical understanding could reduce the cost of entry for new researchers and inventors and lead to spillovers and faster innovation. This could be more important for small-scale consumer facing technologies or products (e.g., cars, refrigerators, windows, food) that may face lower barriers to entry in research compared to, for instance, nuclear power or offshore wind. Second, we also find that innovation focused on consumer performance was incredibly important for this demand side consumer-facing technology and that the improvement mostly relied on integrating external knowledge, suggesting that a joint focus on cost and consumer experience in R\&D is both necessary but also requires a particularly high level of external knowledge integration, and possibly wider hiring practices. This finding also suggests that innovation in demand-side technologies, which have been according to some scholars ‘marginalized’ \cite{Wilson2012}, should be further promoted. A third possible hypothesis to test with future work on other demand side technologies is related to potentially larger contribution of demand-pull policies on cost metrics over time through their impact of scale up and learning by doing compared to R\&D. If corroborated in other technologies, this finding would add to evidence on the major role of regulations, bans and incentives shaping technology costs and trajectories.”

There are several important avenues of future research that are opened up by our analysis. First, future work could expand the cost model by collecting and including data for a broader set of chip architectures. Second, a deeper dive into the role of learning-by-doing is needed both in the cost and performance analysis. Third, building on the work on the physical limits in LED sub-efficiencies, future efforts could focus on identifying priority areas for further performance improvements in LEDs and SSL in general. Fourth, better understanding is needed about the role of demand-side factors, such as policies stimulating market demand for LED lighting (e.g., through incandescent light bulb bans or subsidies for LED adoption), not only in the rapid diffusion of LED-based lighting technologies, which is reasonably well understood by now \cite{Mills2014, Kamat2020, weinold2021quantifying, stegmaier2021incandescent, grubb2021new}, but also in facilitating learning-by-doing, innovation and technology spillovers that have contributed to the technological progress in white LEDs.

\section{Methods}
\label{sec:methods}

\subsection{Multi-Method Approach}

The evolution of white LED device architecture and performance, as well as the progress in understanding the underlying physical phenomena, are relatively well covered in scholarly literature and patents (see Supplementary Note 1 for a brief literature review). However, information provided in such sources is insufficient for our goals on at least three accounts: First, existing work focuses only on selected performance parameters or overall device efficiency, rather than on providing a comprehensive coverage of the whole device sub-efficiencies for a particular LED product or design. Scientific publications also do not always disclose the underlying device architecture or the features responsible for the gains in performance. Second, not all relevant innovations are patented \cite{Pakes_1980,Fontana_2013}. In the case of LED patents in particular, our interviews with industry experts suggest that the propensity to patent is the highest for knowledge related to macroscopic device architecture and chemical composition of phosphors, and the lowest for knowledge related to manufacturing process improvements and microscopic chip architecture that is difficult to reconstruct by reverse engineering. This means that relying only on patent literature would bias results by unduly emphasizing some focus areas and de-emphasizing others. Third, scientific publications and patents typically focus on experimental devices, rather than commercial products. While new LED features, designs and manufacturing methods reported in these sources can potentially result in significant performance gains or cost reductions, it is difficult to ascertain if these improvements have since been adopted in industry. Furthermore, information on LED manufacturing cost and the effect of process improvements on the total cost is highly proprietary. Estimates are occasionally reported in the scientific literature and company publications, but these often do not disclose which parts of the manufacturing process are responsible for the largest contribution to the overall cost, or which improvements led to cost reductions.

To overcome the limitations of existing literature and patent analysis, in this study we rely on a multi-method approach to data collection and analysis, augmenting information obtained from a comprehensive review of the primary scientific literature \cite{Haddaway_2014}, device datasheets, relevant patents, and industry publications with information gained from semi-structured interviews with experts from academia and industry, bottom-up manufacturing cost modelling, and own calculations of LED device sub-efficiencies. We then synthesize information obtained with multiple methods to track the historical progress in white LED technology over time across the three groups of metrics (see Supplementary Method 2) and identify its sources in individual innovations and technology spillovers. We briefly describe our approach to each method below and provide further methodological details in Supplementary Methods. 

\subsection{Comprehensive Literature Review}

We collected data on white LED performance and characteristics in a comprehensive literature review that included scientific publications, patents, conference proceedings from the largest semiconductor and optoelectronics conferences, industry periodicals and roadmaps, as well as company presentations and reports. This review was structured around the three main goals: 1) tracking the evolution of white LED technology over time as indicated by three groups of progress metrics; 2) identifying individual innovations that contributed to this evolution, and quantifying their impact on device performance and manufacturing cost; and 3) determining whether these innovations had originated within the white LED lighting technology domain, or in a field of science or technology outside of SSL, making them involved in technology spillovers.

Relevant sources for the review were found in an iterative search process that involved two components. The first was the search in specialized patent and publication databases as well as company websites. The second component was the analysis of backward citations in the identified sources, starting from the reviews mentioned in Supplementary Note 1 and then iteratively repeating it for all newly identified sources, until no further relevant and significant new sources were found. We also relied on backward citations in these sources for the identification of technology spillovers, considering cited documents as indicators of knowledge origins of an innovation and analyzing whether those documents belonged to the white LED lighting technology domain or not.

\subsection{Semi-Structured Interviews}

To supplement our data collection efforts, verify our findings and identify additional spillovers, we conducted a series of elite semi-structured interviews \cite{tansey2009process} with thirteen eminent experts from academia, industry, and public research sector. Experts were initially selected based on their engagement in different sub-fields of LED research and manufacturing, based on the literature review, then the list was expanded by a ‘snowballing’ tactic based on recommendations from already-interviewed experts. All interviews were conducted between November 2019 and April 2022 by means of video conferencing and lasted for about one hour. An anonymized summary of the background of interviewed experts is provided in Supplementary Method 5. Notably, all our interviewed experts came from Europe or USA, with none representing Asia, which potentially may have biased our findings particularly for the earliest and latest periods of white LED history dominated by LED manufacturers in Japan and China, correspondingly. However, this was not an intentional bias, as none of the identified experts in Asia responded to our interview requests. 

In terms of the interview content, the primary, structured part of the interviews explored which innovations were deemed most relevant to the evolution of device performance, consumer experience and manufacturing cost of LED packages. Thereafter, interviewees were asked to consider the extent to which those innovations may have originated outside of their respective field of expertise and the LED industry more broadly—i.e., which of the innovations potentially involved technology spillovers. The remainder of the interview was focused on learning about particular aspects of the manufacturing processes relevant to cost and performance modelling, the current state of industry, and the circumstances surrounding the innovations and spillovers identified in the first part of the interview. Specific quantitative data was also provided by experts, helping fine-tune the parameters of the manufacturing cost model and verify device performance data.

\subsection{Performance Metrics Calculations}
\label{subsec:performance_metrics}

The contribution of individual technology innovations and spillovers to the progress in overall device efficiency over time is estimated by index decomposition analysis. Mathematically, this involves breaking down a chosen performance indicator into its constituent components, each representing a specific factor that contributes to the change in the indicator \cite{Ang1997}. Specifically, we use the additive logarithmic mean Divisia index method I (LMDI-I), also known as the Additive Sato-Vartia indicator \cite{deBoer2019}. It was developed by Boyd in 1987 \cite{Boyd1987} on the basis of Divisia Index, a method in statistical economics, and subsequently refined.

According to this method, for an overall device efficiency function $F$ that is the product of variables $a, b$ that represent sub-efficiencies, the contribution of the change in a single sub-efficiency variable $a$ between times $t=0$ and $t=T$ can be estimated as \cite{Ang2019}
%
\begin{align}
    \Delta a &= \frac{a_{t=T} - a_{t=0}}{\ln(a_{t=T}) - \ln(a_{t=0})} \times \ln \big ( \frac{a_{t=T}}{a_{t=0}} \big ) \\
    & \stackrel{a_{t=0} \neq a_{t=T}}{=} L(F_{t=T}, F_{t=0}) \times \ln \big ( \frac{a_{t=T}}{a_{t=0}} \big )
\end{align}
%
where $L(F_{t=T}, F_{t=0})$ is the logarithmic mean of $F$ values at times $t=0$ and $t=T$. As we show in Supplementary Method 8, this mean can be approximated in practice, simplifying calculations. These terms contain no residuals, therefore it can be shown that the overall improvement in the device efficiency due to improvements in individual sub-efficiencies is equal to the sum of these improvements in individual sub-efficiencies: 
%
\begin{equation}
    \Delta a + \Delta b  = \Delta F
\end{equation}
%
To document historical improvements in LED device performance accurately, we need data on all sub-efficiencies for the selected device architectures and periods covered. However, the scope of data reporting in the literature is typically limited to selected metrics of interest, rather than the full ensemble of sub-efficiencies that determine the overall device performance. Where gaps in relevant data existed, we filled them with our own performance calculations for individual sub-efficiencies where possible. Specifically, sub-efficiencies related to the emission spectrum of phosphor down-converted LED devices (i.e., spectral efficiency and light conversion efficiency)  were computed from the spectral data, often reported in LED device specifications, using the \texttt{colour-science} package for Python. We also used the same approach to calculate CRI and luminous efficacy of radiation of LED devices (see Supplementary Note 7 for the associated spectral data and calculation results). This approach allowed us to quantify the improvements related to consumer experience and phosphor development in LEDs.

\subsection{Manufacturing Cost Model}

The structure of our bottom-up white LED manufacturing cost model with process-step resolution is generally based on the 2012 \textit{LEDCOM} cost model \cite{ledcomv2}, but we expand it significantly both in scope and in its ability to capture historical trends. The model captures three historical time periods corresponding to different “eras” in white LED manufacturing: the early period of the first high-power white LEDs around 2003, the period of accelerating consumer adoption of LED lighting around 2012, and the most recent period around 2020, the year of our main data collection efforts. For each of these three years, the most prevalent manufacturing equipment was identified through industry periodicals, archived website data from the Internet Archive, and expert interviews. Because the architecture of LED chips has changed significantly since the introduction of the first commercial white LED devices in 1996, three different chip architectures are considered in the model: classical chips, flip chips, and chip-scale package flip chips (see Supplementary Notes 2-3 for the details of each architecture). The details of the manufacturing process for each architecture were collected from the scientific literature, textbooks and relevant patents. In addition, two LED life-cycle analyses were used to validate the model structure and extract some of the necessary quantitative model inputs. These studies captured a large number of white LED manufacturing process steps and included the details on the use of metals, chemicals and electricity for each manufacturing step.

The cost model we developed is based on a cumulative approach to yielded cost \cite{becker2001use}, which we describe in detail in Supplementary Method 7. In this approach, the yielded cost $C_{Y_i}$ of process step $1$ is defined as the ratio between the total cost of step 1 $C_1$ and the yield of step 1 $Y_1$. Thus, for each consecutive step, starting from $i=1$, we get
%
\begin{equation}
    C_{Y_1} = \frac{C_1}{Y_1}, \ C_{Y_2} = C_{Y_{2 \rightarrow 3}} - C_{Y_1} = \frac{C_1(1-Y_2)+C_2}{Y_1Y_2}, \ C_{Y_3}=\dots
\end{equation}
%
Notably, yielded cost per step is dependent on the step order and blind to downstream information \cite{becker2001use}. The yielded cost metric is also cumulative by definition, thus the total cumulative yielded cost is calculated as:
%
\begin{equation}
    \sum_i C_{Y_i} = \frac{\sum_i C_i}{\prod_i Y_i}
\end{equation}
%
The overall outcome of the cost model is thus the cumulative yielded manufacturing cost per LED package for each of the three years considered. In our model, this metric includes all costs associated with producing the chip, including operating costs of the factory. Costs associated with research and development, administrative overhead, or other investment costs are not considered. We note that the purpose of the cost model is not to provide specific estimates of the white LED manufacturing cost for a factory of specific size, geographic location or manufacturing capacity. It is instead intended to capture the impact of specific improvements in the manufacturing process flow on overall cost. It therefore models a hypothetical factory operating the most up-to-date equipment for each model year. Fixed cost parameters are similar to those for a semiconductor factory operating in the United States. Even with these simplifying assumptions, the model reasonably identifies the impact that changes in single process steps can have on the total LED manufacturing cost.

Another important limitation of our cost modelling efforts is that, even though the model captures three different chip architectures in its structure, in the present study we are able to collect, estimate and present the full set of quantitative inputs and outputs only for the classical chip architecture of low-to-mid-power devices. Populating the model with data for the remaining two chip architectures would require additional access to proprietary industry information.

Further details on our manufacturing cost model, including its structure and equations, manufacturing process flows for the chip architectures under consideration, input data, detailed calculations for the yielded costs, as well as the model’s limitations, are provided in Supplementary Method 6.

\subsection{Data Availability}

The manufacturing cost model and associated input data is available are the Zenodo repository with the identifier DOI: 10.5281/ZENODO.8410658. The Python scripts that were used to compute the light quality metrics of \cref{fgr:consumer_experience} are available in the Zenodo repository with the identifier DOI: 10.5281/ZENODO.8410789.

\section*{Corresponding authors}
Correspondence to Michael P. Weinold

\section*{Acknowledgements}

This research was supported by the grant from the Alfred P. Sloan Foundation titled “What factors drive innovation in energy technologies? The role of technology spillovers and government investment”. Michael Weinold gratefully acknowledges support from the Swiss Study Foundation. The authors thank Venkatesh Narayanamurti, Gabriel Chan, Anna Goldstein, Didier Sornette, and participants of the SPIE West 2021 Conference and CEENRG Seminar Series of the University of Cambridge for many helpful discussions and feedback. The authors further express their deepest gratitude to all interviewees for their willingness to participate in this study and share their insights.

\section*{Contributions}
All authors designed the study. M.P.W. and S.K. carried out the main experiments under the supervision of L.D.A. The data was analysed by M.P.W. All authors contributed to the discussion of the results and commented on the manuscript. The draft was written by M.P.W. and S.K.

\section*{Ethics Declarations}
The authors declare no competing interests.

\newpage
\printbibliography

\end{document}